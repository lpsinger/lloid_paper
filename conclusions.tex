\section{Conclusions}
\label{SECV}\label{sec:conclusions}

\editorial{We should drive home the point that our method is as fast as the fft convolution but without any latency at all.}%
We have demonstrated a computationally feasible procedure for the rapid and even preemptive
detection of gravitational waves emitted during the coalescence of neutron
stars and stellar-mass black holes. Our method is as fast as standard \fft{}
convolutions but without any latency at all. Our search targets are expected to
produce prompt electromagnetic signals and may be the progenitors of some short
hard gamma-ray bursts.  Rapid alerts to the broader astronomical community will
improve the chances of detecting an electromagnetic counterpart in bands from
gamma-rays down to radio.  We anticipate requiring $\sim$\numcpus\ modern
%
\editorial{We need to do this calculation using the \flops\ counts and the
number of templates. CHAD: Agreed, but if you do it by the books it seems
misleading.  I have adjusted it to 600.  This number is taken from assuming
that 1 657 template sub-bank can be filtered on one core and dividing the total
number of templates (10$^5$) by 657 which gives you 150 cores per detector. I
put it in section 2. }%
%
computer cores to analyze a four-detector network of gravitational-wave data
for binary neutron stars and stellar-mass black holes.  This is well within the
current computing capabilities of the \textsc{lsc} Data Grid~\cite{LDG}.

The algorithm we described has no intrinsic latency.  However, there are
fundamental and practical latencies associated with the analysis and detection
procedure. For example, the \LIGO{} detectors, data acquisition is synchronized
to a 1/(16~Hz) cadence introducing an up-front latency.  Data
aggregation from the observatories will travel over various networks, each
capable of high bandwith but perhaps only modest latency.  This could amount to
a similar latency of $\sim$100 ms.  Lastly, unless a real-time infrastructure is
adopted post data acquisition, it is likely that there will be an inherent
latency introduced by such infrastructure.  We have shown a prototype
implementation using \gstlal\ that is capable of $\sim$1 s latency. In our
opinion, significant work would have to be done in order to improve upon this
number. For example, a tighter integration of analysis and data acquisition
would be beneficial. This should be considered for third-generation detector
design.


\editorial{The way this is worded, it sounds like a big omission.}%
We have omitted discussion of source localization though point
the reader to some theoretical estimates \SNR~\cite{Fairhurst2009}.
%However, one should not immediately dismiss the practical usefulness of a
%poorly localized source. Even with poor localization, it should be possible to
%begin downselecting what observatories could view a potential signal and for
%such observatories to begin any necessary prerequisite activities. 
In future works we will explore more rigorously the pointing prospects with
realistic simulations using the infrastructure and techniques described here.
\editorial{nvf: It's a bit pie-in-the-sky, but we could also mention reconfiguring
the signal recycling mirror to optimize SNR at merger. There's a lot of science there.}

%Latency budget, including `before' and `after' quotes for:

%\begin{itemize}
%\item Data acquisition
%\item Calibration
%\item Data aggregation
%\item Analysis
%\item Localization
%\item Alert
%\item Telescope actuation
%\item Total
%\end{itemize}

%Future work:

%\begin{itemize}
%\item Sub-solar mass search
%\item Hierarchical detection
%\end{itemize}
