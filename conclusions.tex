\section{Conclusions}
\label{sec:conclusions}

We have demonstrated a computationally feasible procedure for the rapid and
even early-warning detection of \GW{}s emitted during the coalescence
of neutron stars and stellar-mass black holes. Our method is as fast as
standard \fft{} convolutions but allows for latency-free, real-time
operation.  Our search targets are expected to produce prompt electromagnetic
signals and may be the progenitors of some short hard gamma-ray bursts.  Rapid
alerts to the broader astronomical community will improve the chances of
detecting an electromagnetic counterpart in bands from gamma-rays down to
radio.  We anticipate requiring $\sim 4 \times 10 = 40$ modern multi-core computers to
search for binary neutron stars using coincident \GW\ data from a four-detector network.
This is well within the current computing capabilities of the \LIGO{} Data Grid%
\footnote{\url{https://www.lsc-group.phys.uwm.edu/lscdatagrid/}}. In the future, this
could be improved upon further by conditionally reconstructing the \SNR{} time-series
only during times when a composite detection statistic crosses a
threshold~\citep{svd-compdetstat}.

The algorithm we described has no intrinsic latency.  However, there are
fundamental and practical latencies associated with the analysis and detection
procedure. For example, the \LIGO{} data acquisition records science data in 1/16~s
blocks~\citep{Bork2001}. Data must also be aggregated from all of the \GW\
observatories using global computer networks capable of high bandwidth but perhaps only
modest latency.  This could introduce a latency of $\sim$100~ms.  Lastly, unless a
real-time infrastructure is adopted post data acquisition, it is likely that there will be 
an inherent latency introduced by the software architecture itself.  We have shown 
prototype implementation of \lloid{} using the open source signal processing software
\gstreamer\ and \gstlal. Significant work would have to be done in order to
improve upon the latency of our implementation, including tighter integration between
data acquisition and analysis. This should be considered for third-generation detector
design.

Although we have demonstrated a feasible method for advance detection we have
not explored the accuracy of sky localization that is possible before merger.
\citet{Fairhurst2009} discusses some of the theoretical prospects for early
sky localization.  Our future work will explore the prospects of early-warning
detection with realistic simulations of binary mergers using the infrastructure
and techniques described here. 

\editorial{nvf: It's a bit pie-in-the-sky, but we could also mention
reconfiguring the signal recycling mirror to optimize SNR at merger. There's a
lot of science there.}

%Latency budget, including `before' and `after' quotes for:

%\begin{itemize}
%\item Data acquisition
%\item Calibration
%\item Data aggregation
%\item Analysis
%\item Localization
%\item Alert
%\item Telescope actuation
%\item Total
%\end{itemize}

%Future work:

%\begin{itemize}
%\item Sub-solar mass search
%\item Hierarchical detection
%\end{itemize}
