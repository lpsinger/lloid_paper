\section{Conclusions}
\label{sec:conclusions}

We have demonstrated a computationally feasible filtering algorithm for the rapid and
even early-warning detection of \GW{}s emitted during the coalescence
of neutron stars and stellar-mass black holes.  It is one part of a complicated
analysis and observation strategy that will unfortunately have other sources of
latency.  However, we hope that it will motivate further work to reduce such
technical sources of \GW{} observation latency and encourage the possibility of
even more rapid \EM\ followup observations to catch prompt emission in the
advanced detector era.  

The \lloid{} method is as fast as standard \fft{} convolutions but allows for
nearly latency-free, real-time operation.  We anticipate requiring $\sim$100
modern multi-core computers to search for binary neutron stars using
coincident \GW\ data from a multi-detector network.  In the future, additional
computational savings could be acheived by conditionally reconstructing the
\SNR{} time-series only during times when a composite detection statistic
crosses a threshold~\citep{svd-compdetstat}.  However, the anticipated required
number of computers is well within the current computing capabilities of the
\LIGO{} Data Grid.

\CBC\ events may be the progenitors of some short hard \GRB{}s and are expected
to be accompanied by prompt \EM{} signals.  Rapid alerts to the broader
astronomical community will improve the chances of detecting an \EM{}
counterpart in bands from gamma-rays down to radio.  It will be extremely
challenging for all but the rarest events to localize the source well enough in
advance to detect the onset of prompt emission, however localization prior to
the onset of extended emission on the 10--100~s time-scale is more easily
achieved and will benefit from the full bandwidth and \SNR\ of the \GW{} event
at merger.  Probing even the extended emission timescale requires very low
latency filtering schemes such as the one described here.

We have shown a prototype implementation of the \lloid{} algorithm using
\gstreamer, an open source signal processing platform.  Work will be done in
order to improve upon the latency of our implementation.  Ultimately the best
possible latency would be achieved by tighter integration between data
acquisition and analysis with dedicated hardware and software. This could be
considered for third-generation detector design.  Also possible for
third-generation instruments, the \lloid{} method could provide the input for a
dynamic tuning of detector response via the signal recycling mirror to match
the frequency of maximum sensitivity to the instantaneous frequency of the
\GW{} waveform.  This is a challenging technique, but it has the potential for
substantial gains in \SNR{} and timing accuracy \citep{PhysRevD.47.2184}.

Although we have demonstrated a computationally feasible detection statistic
for advance detection, we have not yet explored data calibration and whitening,
triggering, coincidence, and ranking of gravitational wave candidates in a
framework that supports early \EM\ followup.
\begin{comment}
One complication in deploying a
low-latency or early-warning search is that whitening the detector data
involves estimating the noise power spectrum, which is often done acausally.
We will have to look into causal or low latency alternatives to conventional
spectral estimation techniques.  
\end{comment}
We will explore these and also using the time-slice decomposition and the \SVD\
to form low-latency signal-based vetoes (e.g.~$\chi^2$~statistics) that have
been essential for glitch rejection used in previous \GW{} \CBC{} searches.

Our future work must more deeply address sky localization accuracy in a
realistic setting as well as observing strategies. Here, we have followed
\citet{Fairhurst2009} in estimating the area of 90\% localization confidence in
terms of timing uncertainties alone, but it would be advantageous to use a
galaxy catalog to inform the telescope tiling \citep{galaxy-catalog}. Because
early detections will arise from nearby sources, the galaxy catalog technique
might be an important ingredient in reducing the fraction of sky that must be
imaged.  Extensive simulation campaigns incorporating realistic binary merger
rates and detector networks will be necessary in order to fully understand the
prospects for early-warning detection, localization, and \EM\ followup using
the techniques we have described.
