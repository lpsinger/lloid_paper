\section{Conclusions}
\label{sec:conclusions}

We have demonstrated a computationally feasible procedure for the rapid and
even early-warning detection of \GW{}s emitted during the coalescence of
neutron stars and stellar-mass black holes. Our method is as fast as standard
\fft{} convolutions but allows for latency-free, real-time operation.  We
anticipate requiring $\sim 4 \times 10 = 40$ modern multi-core computers to
search for binary neutron stars using coincident \GW\ data from a four-detector
network.  In the future, additional computational savings could be acheived by
conditionally reconstructing the \SNR{} time-series only during times when a
composite detection statistic crosses a threshold~\citep{svd-compdetstat}.
However, the anticipated required number of computers is well within the
current computing capabilities of the \LIGO{} Data
Grid\footnote{\url{https://www.lsc-group.phys.uwm.edu/lscdatagrid/}}.

\CBC\ events may be the progenitors of some short hard \GRB{}s and are expected
to be accompanied prompt \EM{} signals.  Rapid alerts to the broader
astronomical community will improve the chances of detecting an \EM{}
counterpart in bands from gamma-rays down to radio.

The detection algorithm we described has no intrinsic latency.  However, there
are fundamental and practical latencies associated with the analysis and
detection procedure. For example, the \LIGO{} data acquisition records science
data in 1/16~s blocks~\citep{Bork2001}. Data must also be aggregated from all
of the \GW\ observatories using global computer networks capable of high
bandwidth but perhaps only modest latency.  This could introduce a latency of
$\sim$100~ms.  Lastly, the software implementation of the algorithm itself may
introduced latency.  We have shown prototype implementation of \lloid{} using
the open source signal processing platform \gstreamer. Significant work would
have to be done in order to improve upon the latency of our implementation,
including tighter integration between data acquisition and analysis. This could
be considered for third-generation detector design.  Also possible for
third-generation instruments, the \lloid{} method could provide the input for a
dynamic tuning of detector response via the signal recycling mirror to match
the frequency of maximum sensitivity to the instantaneous frequency of the
\GW{} waveform.  This is a challenging technique, but it has the potential for
vast gains in \SNR{} and timing accuracy \citep{PhysRevD.47.2184}.

Although we have demonstrated a feasible method for advance detection
\emph{filtering}, we have not yet explored data whitening, triggering,
coincidence, and ranking of gravitational wave candidates in a framework that
supports early \EM\ followup.
% Drew: I don't understand what is trying to be said here.  Time-slices have no
% cross-correlation with eachother since they are orthogonal.
\begin{comment}If adjacent time slices have non-negligible cross-correlation
with each other, then there may be some subtleties in the design of these
postprocessing stages that we must address in order to translate the
early-warning outputs into candidate events.\end{comment}
%
However, we believe these obstacles are surmountable.
% Drew: Do we really want to say this. It doesn't really address any of the
% issues above.
\begin{comment}For an example of how the last obstacle could be approached, the
time-slice decomposition and the \SVD\ will help us form low-latency
signal-based vetoes (for example, a $\chi^2$ statistic) that have been
essential for glitch rejection used in previous \GW{} \CBC{}
searches.\end{comment}

Our future work must more deeply address sky localization accuracy prior to
merger as well as observing strategies. Here, we have followed
\citet{Fairhurst2009} in estimating the area of 90\% localization confidence in
terms of timing uncertainties alone, but it would be advantageous to use a
galaxy catalog to inform the telescope tiling \citep{galaxy-catalog}. Because
early detections will arise from nearby sources, the galaxy catalog technique
might be an important ingredient in reducing the fraction of sky that must be
imaged.  Extensive simulation campaigns incorporating realistic binary merger
rates and detector networks will be necessary in order to fully understand the
prospects for early-warning detection, localization, and \EM\ folluwup using
the techniques we have described.


%Latency budget, including `before' and `after' quotes for:

%\begin{itemize}
%\item Data acquisition
%\item Calibration
%\item Data aggregation
%\item Analysis
%\item Localization
%\item Alert
%\item Telescope actuation
%\item Total
%\end{itemize}
