\section{Conclusions}
\label{SECV}\label{sec:conclusions}

We have demonstrated a computationally feasible procedure for the rapid and
even preemptive detection of \GW{}s emitted during the coalescence
of neutron stars and stellar-mass black holes. Our method is as fast as
standard \fft{} convolutions but allows for zero latency, sample-in-sample-out
operation.  Our search targets are expected to produce prompt electromagnetic
signals and may be the progenitors of some short hard gamma-ray bursts.  Rapid
alerts to the broader astronomical community will improve the chances of
detecting an electromagnetic counterpart in bands from gamma-rays down to
radio.  We anticipate requiring $\sim100$ modern multi-core computers to
analyze a four-detector network of \GW{} data for binary neutron
stars and stellar-mass black holes.  This is well within the current computing
capabilities of the \LIGO{} Data Grid~\cite{LDG}. In the future, this could be
improved upon further by conditionally reconstructing the \SNR{} time-series
only during times when a composite detection statistic crosses a
threshold~\cite{svd-compdetstat}.

The algorithm we described has no intrinsic latency.  However, there are
fundamental and practical latencies associated with the analysis and detection
procedure. For example, the \LIGO{} detectors, data acquisition is synchronized
to a 1/(16~Hz) cadence, introducing an up-front latency~\cite{Bork2001}. Data
aggregation from the observatories will travel over various networks, each
capable of high bandwidth but perhaps only modest latency.  This could amount to
a similar latency of $\sim$100 ms.  Lastly, unless a real-time infrastructure
is adopted post data acquisition, it is likely that there will be an inherent
latency introduced by such infrastructure.  We have shown a prototype
implementation of \lloid{} using the open source signal processing software
\gstreamer\ and \gstlal. Significant work would have to be done in order to
improve upon the latency capability of our implementation, for example, more
tightly integrating analysis and data acquisition. This should be considered
for third-generation detector design.

Although we have demonstrated a feasible method for advance detection we have
not explored the accuracy of sky localization that is possible before merger.
Ref.~\cite{Fairhurst2009} discusses some of the theoretical prospects for early
sky localization.  Our future work will explore the prospects of early-warning
detection with realistic simulations of binary mergers using the infrastructure
and techniques described here. 

\editorial{nvf: It's a bit pie-in-the-sky, but we could also mention
reconfiguring the signal recycling mirror to optimize SNR at merger. There's a
lot of science there.}

%Latency budget, including `before' and `after' quotes for:

%\begin{itemize}
%\item Data acquisition
%\item Calibration
%\item Data aggregation
%\item Analysis
%\item Localization
%\item Alert
%\item Telescope actuation
%\item Total
%\end{itemize}

%Future work:

%\begin{itemize}
%\item Sub-solar mass search
%\item Hierarchical detection
%\end{itemize}
