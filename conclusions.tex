\section{Conclusions}
\label{SECV}\label{sec:conclusions}

We have demonstrated a computationally feasible procedure for the rapid
detection of gravitational waves emitted during the coalescence of neutron
stars and stellar-mass black holes.  These sources are expected to produce
prompt electromagnetic signals and may be the progenitors of some short hard
gamma-ray bursts.  Rapid alerts to the broader astronomical community may
improve the chances of detecting an electromagnetic counterpart in bands from
X-ray down to radio.  We antipate requiring no more than \numcpus\ modern
computer cores to analyze a four-detector network of gravitational-wave data
for binary neutron stars and stellar mass black holes.  This is within the
current computing capabilities of the \textsc{lsc} Data Grid~\cite{LDG}.

The algorithm we described has no intrinsic latency.  However, there are
fundamental and practical latencies associated with the analysis and detection
procedure. For example, the \LIGO{} detectors, data acquisition is synchronized
to a 1/16 Hz cadence introducing an up-front latency of 125 ms.  Data
aggregation from the observatories will travel over various networks, each
capable of high bandwith but perhaps only modest latency.  This could amount to
a similar latency of $\sim$ 100 ms.  Lastly, unless a realtime infrastructure is
adopted post data acquisition, it is likely that there will be an inherent
latency introduced by such infrastructure.  We have shown a prototype
implementation using \gstlal\ that is capable of $\sim$1 s latency. In our
opinion, significant work would have to be done in order to improve upon this
number. However, it should be considered for third generation detector design.
For example, a tighter integration of analysis and data acquisition would be
beneficial.

We have omitted discussion of source localization. Localization is known to
be poor for signals of low \SNR~\cite{Fairhurst2009}.
However, one should not immediately dismiss the practical usefulness of a
poorly localized source. Even with poor localization, it should be possible to
begin downselecting what observatories could view a potential signal and for
such observatories to begin any necessary prerequisite activities. In future
works we wil explore more rigorously the pointing prospects with realistic
simulations using the infrastructure and techniques described in this work.

%Latency budget, including `before' and `after' quotes for:

%\begin{itemize}
%\item Data acquisition
%\item Calibration
%\item Data aggregation
%\item Analysis
%\item Localization
%\item Alert
%\item Telescope actuation
%\item Total
%\end{itemize}

%Future work:

%\begin{itemize}
%\item Sub-solar mass search
%\item Hierarchical detection
%\end{itemize}
