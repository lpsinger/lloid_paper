\section{Conclusions}
\label{sec:conclusions}

We have demonstrated a computationally feasible procedure for the rapid and
even early-warning detection of \GW{}s emitted during the coalescence
of neutron stars and stellar-mass black holes. Our method is as fast as
standard \fft{} convolutions but allows for latency-free, real-time
operation.  We anticipate requiring $\sim 4 \times 10 = 40$ modern multi-core computers to
search for binary neutron stars using coincident \GW\ data from a four-detector network.
This is well within the current computing capabilities of the \LIGO{} Data Grid%
\footnote{\url{https://www.lsc-group.phys.uwm.edu/lscdatagrid/}}. In the future, this
could be improved upon further by conditionally reconstructing the \SNR{} time-series
only during times when a composite detection statistic crosses a
threshold~\citep{svd-compdetstat}.

\CBC\ events may be the progenitors of some short hard gamma-ray bursts and are
expected to be accompanied prompt electromagnetic signals.  Rapid
alerts to the broader astronomical community will improve the chances of
detecting an electromagnetic counterpart in bands from gamma-rays down to
radio.  The detection algorithm we described has no intrinsic latency.  However, there are
fundamental and practical latencies associated with the analysis and detection
procedure. For example, the \LIGO{} data acquisition records science data in 1/16~s
blocks~\citep{Bork2001}. Data must also be aggregated from all of the \GW\
observatories using global computer networks capable of high bandwidth but perhaps only
modest latency.  This could introduce a latency of $\sim$100~ms.  Lastly, unless a
real-time infrastructure is adopted post data acquisition, it is likely that there will be 
an inherent latency introduced by the software architecture itself.  We have shown 
prototype implementation of \lloid{} using the open source signal processing software
\gstreamer\ and \gstlal. Significant work would have to be done in order to
improve upon the latency of our implementation, including tighter integration between
data acquisition and analysis. This should be considered for third-generation detector
design.

Although we have demonstrated a feasible method for advance detection we have
not yet explored triggering, coincidence, and ranking of gravitational wave
candidates in a framework that supports early \EM\ followup.  If adjacent time slices
have non-negligible cross-correlation with each other, then there may be some
subtleties in the design of these postprocessing stages that we must address in order
to translate the early-warning outputs into candidate events.

Our future work must also address sky localization accuracy prior to merger as well as
observing strategies.  \citep{Fairhurst2009} derives the area of 90\% localization
confidence in terms of timing uncertainties, but it would be advantageous to use
a galaxy catalog to inform the telescope tiling \citep{galaxy-catalog}.  Because
early detection is likely to be especially effective for nearby sources, the galaxy
catalog technique might be an important ingredient in reducing the fraction of
sky that must be imaged.  Extensive injection campaigns and simulations
incorporating realistic binary merger rates will be necessary in order to fully
understand the prospects for early-warning detection, localization, and \EM\ folluwup
using the techniques we have described.

\editorial{nvf: It's a bit pie-in-the-sky, but we could also mention
reconfiguring the signal recycling mirror to optimize SNR at merger. There's a
lot of science there.}

%Latency budget, including `before' and `after' quotes for:

%\begin{itemize}
%\item Data acquisition
%\item Calibration
%\item Data aggregation
%\item Analysis
%\item Localization
%\item Alert
%\item Telescope actuation
%\item Total
%\end{itemize}
