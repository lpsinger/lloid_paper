\documentclass[aps,prd,showpacs,groupedaddress,showkeys,preprintnumbers]{revtex4}
\usepackage{hyperref}
% \usepackage{amsmath} % Incompatible with iopart
\usepackage{amssymb}
\usepackage{array}
\usepackage{verbatim}
\usepackage{graphicx}
\graphicspath{{figures/}}
\usepackage{subfig}
\usepackage[usenames,dvipsnames]{color}
\usepackage[normalem]{ulem}
%\usepackage[bookmarksnumbered, bookmarksopen, breaklinks, colorlinks]{hyperref} % Breaks my build - Antony

\newcommand{\Msun}{\ensuremath{M_{\odot}}}
% Editing macros

\newcommand\citeneeded{\textsc{\color{blue}[citation needed]}}
% Borrowed from http://albert.rierol.net/latex_tips.html
\newcommand\editorial[1]{\mbox{}\marginpar{\footnotesize\raggedright\hspace{0pt}\color{blue}\emph{#1}}}

\newcommand{\be}{\begin{equation}}
\newcommand{\ee}{\end{equation}}
\newcommand{\ben}{$$}
\newcommand{\een}{$$}
\newcommand{\bea}{\begin{eqnarray}}
\newcommand{\eea}{\end{eqnarray}}
\newcommand{\bean}{\begin{eqnarray*}}
\newcommand{\eean}{\end{eqnarray*}}
%\newcommand{\e}{{\rm e}} % breaks my build
%\newcommand{\tr}{{\rm tr}} % breaks my build
\newcommand{\erf}{{\rm erf}}
\newcommand{\ie}{{\rm i.e.}}
\newcommand{\n}[1]{\label{#1}}
\newcommand{\ind}[1]{\mbox{\tiny{#1}}}
\newcommand{\nn}{\nonumber \\ \nonumber \\}
\newcommand{\Fp}{{\vec{F}}^{+}}
\newcommand{\Fc}{{\vec{F}}^{\times}}
\newcommand{\Vp}{{\vec{V}}^{+}}
\newcommand{\Vc}{{\vec{V}}^{\times}}
\newcommand{\abs}[1]{\lvert#1\rvert}
\newcommand{\norm}[1]{\lVert#1\rVert}
%\renewcommand{\vec}[1]{\mbox{\boldmath$#1$}}

% Graham's macros
\renewcommand{\vec}[1]{\mathbf{#1}}
\newcommand{\tran}[1]{ #1^{\rm T}}
%\newcommand{\abs}[1]{\left\vert#1\right\vert}
\newcommand{\fp}{F^+}
\newcommand{\fc}{F^\times}
\newcommand{\vfp}{\vec{F}^+}
\newcommand{\vfc}{\vec{F}^\times}
\newcommand{\vfph}{\hat{\vec{F}}^+}
\newcommand{\vfch}{\hat{\vec{F}}^\times}
\newcommand{\fps}{F^{+2}}
\newcommand{\fcs}{F^{\times2}}
\newcommand{\hp}{h_+}
\newcommand{\hc}{h_\times}
\newcommand{\sh}{\sigma_h}
\newcommand{\sn}{\sigma_n}
\newcommand{\shs}{\sigma_h^2}
\newcommand{\sns}{\sigma_n^2}
\newcommand{\vkh}{\hat{\vec{k}}}
\newcommand{\C}{\vec{C}}
\newcommand{\CI}{\vec{C}^{-1}}
\newcommand{\dC}{\text{det}\,\C}



\begin{document}

\title{(OUTLINE DRAFT)\\ Low-latency search for gravitational-waves: prospects for near real time detection of compact binary coalescence}

\date{\today}

\author{Kipp Cannon}
% \email{antony.searle@anu.edu.au}
\affiliation{The LIGO Laboratory - California Institute of Technology, Pasadena, CA 91125}

\author{Adrian Chapman}
% \email{antony.searle@anu.edu.au}
\affiliation{The LIGO Laboratory - California Institute of Technology, Pasadena, CA 91125}

\author{Chad Hanna}
% \email{antony.searle@anu.edu.au}
\affiliation{The LIGO Laboratory - California Institute of Technology, Pasadena, CA 91125}

\author{Drew Keppel}
% \email{antony.searle@anu.edu.au}
\affiliation{The LIGO Laboratory - California Institute of Technology, Pasadena, CA 91125}

\author{Antony C. Searle}
% \email{antony.searle@anu.edu.au}
\affiliation{The LIGO Laboratory - California Institute of Technology, Pasadena, CA 91125}

\author{Alan J. Weinstein}
% \email{antony.searle@anu.edu.au}
\affiliation{The LIGO Laboratory - California Institute of Technology, Pasadena, CA 91125}


\begin{abstract}
The future of gravitational-wave astronomy permits the possibility that 
gravitational-wave detections from inspiralling compact objects could direct
electromagnetic followup searches. In order to be practical the latency of
gravitational-wave searches informing other astronomical instruments 
would have to be small to catch the peak electromagnetic radiation output. 
It is desirable to consider time domain filters for gravitational-wave data
to reduce the latency.   Typically the cost of time domain convolutions 
is prohibitive for any realizable analysis.  Here we present a method to
make the cost of time domain convolutions reasonable with the hope that it
could be employed for near real time detection of gravitational-wave signals
from compact binary mergers.  
\end{abstract}

\pacs{}

\keywords{Gravitational Waves, Laser Interferometry}

\preprint{}

\maketitle

\begin{enumerate}
\item{Introduction}
\begin{itemize}
\item{Introduce how GW searches could direct EM searches and do literature review}
\item{Introduce need for low latency searches and setup the discussion of our
method as well as a literature review for similar ideas}
\item{Outline the rest of the paper}
\end{itemize}
\item{Method}
\begin{itemize}
\item{Describe SVD decomposition of template banks}
\item{Describe selective downsampling of waveform portions}
\item{Describe computational cost / latency of analysis and compare to the 
standard method (theoretical)}
\end{itemize}
\item{Analysis}
\begin{itemize}
\item{Describe the input to the simulations we will do, i.e. the template bank,
the match conditions, the whitening, all the gory details...}
\end{itemize}
\item{Results}
\begin{itemize}
\item{Show results of our ability to reconstruct waveforms and compare to 
``theory"}
\item{Show results of the latency and computational costs and compare to ``theory"}
\end{itemize}
\item{Conclusions}
\begin{itemize}
\item{Sum up the status and describe the future work}
\end{itemize}
\end{enumerate}
\begin{acknowledgments}
LIGO was constructed by the California Institute of Technology and Massachusetts Institute of Technology with funding from the National Science Foundation and operates under cooperative agreement PHY-0107417. This paper has LIGO Document Number LIGO-P0900004-v1.
\end{acknowledgments}


\end{document}


