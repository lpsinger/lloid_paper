%%%%%%%%%%%%%%%%%%%%%%%%%%%%%%%%%%%%%%%%%%%%%%%%%%%%%%%%%%%%%%%%%%%%%%%%%%%%%%%
%% Si vous voulez plus petit passez en 11pt voire encore plus petit en 10pt
\documentclass[11pt]{report}
\author{Romain Cariou}
\date{Juin 2009}
\title{rapport r�sum�}

\usepackage{floatrow}


%Symboles divers dont euro ...
\usepackage{marvosym}

%% Je suis francophone !
\usepackage[francais]{babel}\usepackage[T1]{fontenc}

%% Je veux utiliser n�anmoins des fontes qui � paraissent bien � en PDF
\usepackage[cyr]{aeguill}

%% J'aime bien pouvoir contr�ler mes hauts de page !
\usepackage{fancyhdr}

\fancyhead[R]{\leftmark} % Nom du chapitre en cours

    
%% Je veux pouvoir inclure des figures...
%% � commenter si vous voulez faire du DVI :
\usepackage[pdftex]{graphicx}

%% ... des figures ``jpeg'' ou ``pdf''
%% � commenter si vous voulez faire du DVI :
\DeclareGraphicsExtensions{.jpg,.pdf,.png}

%%position des images un peu partout
\usepackage{wrapfig}

%% Je veux cr�er des Hyperdocuments
%% � commenter si vous voulez faire du DVI :
\usepackage[pdftex,colorlinks=true,linkcolor=blue,citecolor=blue,urlcolor=blue]{hyperref}

%% Je contr�le la taille de ma zone imprim�e...
\usepackage{anysize}
%% ...en d�finissants les marges {gauche}{droite}{haute}{basse}
\marginsize{16mm}{16mm}{-10mm}{5mm}

%pour la numerotation des pages
\usepackage{lastpage}
\makeatletter
\renewcommand{\@evenfoot}%
	{\hfil \upshape {\thepage} / \pageref{LastPage}}
\renewcommand{\@oddfoot}{\@evenfoot}
\makeatother

%pour changer le style des notes de bas de page
\renewcommand{\thefootnote}{\fnsymbol{footnote}}


%pour changer les marges d'une page
\usepackage{chngpage}

% Usage:
% ------
%
%%%%%%%%%%%%%
% \changetext
%%%%%%%%%%%%%
%     The \changetext command is for changing the size and horizontal position
% of the text block on a page. The command takes 5 arguments, each of which
% is a length or is empty. i.e.,
%
% \changetext{textheight}{textwidth}{evensidemargin}{oddsidemargin}{columnsep}
%
% The given lengths are added to the corresponding current lengths and
% the remainder of the current page is typeset using the changed text block
% layout. The new layout remains in effect until another \change... command
% is issued.
%
%%%%%%%%%%%%%
% \changepage
%%%%%%%%%%%%%
%     The \changepage command is for changing the general layout of
% a page. The command takes 9 arguments, each of which is a length or is empty.
% The first 5 arguments are the same as for \changetext and have the same effect.
% The last four arguments are:
%
% \changepage{5 args}{topmargin}{headheight}{headsep}{footskip}
%
% These lengths are added to the corresponding current lengths and
% thus modify the vertical positions of the elements of the page. The
% remainder of the current page is typeset using the changed text block
% and page layout. The new layout remains in effect until another
% \change... command is issued.




%pour mettre quelquechose en filigrane
\usepackage{tikz}

%modifier les marges
\usepackage[francais]{layout}

%pour ecrire sur 2 colonnes
\usepackage{multicol}
\setlength{\columnseprule}{0pt} %largeur trait s�parateur
\setlength{\columnsep}{22pt}

%pour ecrire a cot� des images
\usepackage{floatflt}


\newenvironment{narrow}[2]{%
\begin{list}{}{%
\setlength{\topsep}{0pt}%
\setlength{\leftmargin}{#1}%
\setlength{\rightmargin}{#1}%
\setlength{\listparindent}{\parindent}%
\setlength{\itemindent}{\parindent}%
\setlength{\parsep}{\parskip}}%
\item[]}{\end{list}}

%%%%%%%%%%%%%%%%%%%%%%%%%%%%%%%%%%%%%%%%%%%%%%%%%%%%%%%%%%%%%%%%%%%%%%%%%%%%%%%
%% Je peux d�finir mes propres � commandes �...
%Pour changer le titre de la bibliographie
\renewcommand{\bibname}{R\'ef\'erences}

%% pour citer une figure :
\newcommand{\figref}[1]{figure~\ref{#1}}

%% pour citer une �quation :
\newcommand{\Ref}[1]{(\ref{#1})}

%% pour mettre l'emphase sur un mot ou un groupe de mots :
\newcommand{\empha}[1]{\textit{\textbf{#1}}}

%% op�rateurs de d�rivation :
\newcommand{\D}{\partial}
\newcommand{\Dt}{\partial_t}
\newcommand{\Dx}{\partial_x}
\newcommand{\Dy}{\partial_y}

\begin{document}
%filigrane
\begin{tikzpicture}[remember picture,overlay]
	\node[rotate=60,scale=15,text opacity=0.1]
	  at (current page.center) {Confidential};
\end{tikzpicture}


%pour changer le titre de la Bibliographie
\renewcommand{\bibname}{R\'ef\'erences}

\author{Romain Cariou}
\title{Fiche r�sum� d�but de stage}


%% Voil� mes l�gendes de figures comme je les aime
\makeatletter
\def\figurename{{\protect\sc \protect\small\bfseries Fig.}}
\def\f@ffrench{\protect\figurename\space{\protect\small\bf \thefigure}\space}
\let\fnum@figure\f@ffrench%
\let\captionORI\caption
\def\caption#1{\captionORI{\rm\small #1}}
\makeatother


%Pour les numerotations
\renewcommand{\thesection}{\Roman{section}}
\renewcommand{\thesubsection}{\thesection .\Alph{subsection}}


%%%%%%%%%%%%%%%%%%%%%%%%%%%%%%%%%%%%%%%%%%%%%%%%%%%%%%%%%% Couverture :
%\thispagestyle{plain}
{\Large

\begin{center}
%\vskip1cm
%% Le \vphantom{\int_\int} sert � introduire de l'espace entre les deux lignes
%% (essayez donc de le commenter)
\textbf{{\LARGE Bias in geometric sample mean}}
%$ \vphantom{\int_\int}$
 %\\
%\textbf{{\LARGE de l'exp�rience LIGO : Lloid}}
\begin{center}
\rule[1ex]{5cm}{0.15ex}
\end{center}
\end{center}
\vspace*{-0.8cm}
\begin{center}
\textsl{ Romain Cariou }
\end{center}



\normalsize
\vspace*{0.5cm}
\begin{multicols}{2}


Here we compute the bias factor $\beta$ of the sample median of a chi-square distributed variables relative to the sample geometric mean.\\ Let's define n chi-square distributed variable : $x_{l}$ l$\in\left[\hspace{-1ex}\left[\hspace{0.5ex}1,n\right]\kern-0.15em\right]$. The probability density function for each $x_{l}$ is given by : $f(x)=\frac{1}{2} e^{-x/2}$, where $$2=\langle x \rangle = \int_{0}^{\infty}xf(x)dx$$ is the \textit{expected value} of the random variable $x$. The median, $x_{1/2}$ is defined by $$ \frac{1}{2}=\int_{0}^{x_{1/2}}f(x)dx$$ which yields $x_{1/2}=2\,ln2$. Thus the bias of the \textit{expected value} is $ln2$.\\
Let's $\rho$ be the geometric sample mean of the chi-square variables : $$\rho=\left\langle(\prod_{l=1}^{n}x_{l})^{1/n}\right\rangle=\left\langle exp \left[\frac{1}{n}\sum_{l=1}^{n} \ln x_{l}\right] \right\rangle$$ so $\ln \rho$ is the arithmetic mean from the logarithm of the discret distribution and thus, for a continuous variable as $x_{l}$ we have : $$ \rho=exp \left(\int_{0}^{\infty}\ln x\,f(x)\,dx \right)$$ $$=exp \left(\frac{1}{2} \int_{0}^{\infty}e^{-x/2}\,\ln x dx \right)$$
The integral above can be calculated\footnote{\label{1}\textit{Table of integrals, series and products}, I.S. Gradshteyn/I.M. Ryzhik, BI ((256))(2).} and we find : $$ \rho\,\,=\,\frac{2}{e^{C}}= 1.122918967...\,. $$ where $C\,= \lim\limits_{s\to \infty}\left(\sum_{m=1}^{s}\frac{1}{m}-\ln s \right)$ $= 0.577215...\,.$\\
As the expected value of uncorrelated random variable is the product of the expected values (E(XY)=E(X)E(Y)), we can calculate the expected value of the geometric sample mean :$\langle \rho \rangle = \langle x_{l}^{1/n} \rangle^{n}$. The probability density function of the new random variable $Y=x^{1/n}$ is : $$P(Y)=\frac{n}{2}.e^{-Y^{n}/2}.Y^{n-1}$$ and so $$
\langle \rho \rangle = \left[\frac{n}{2}\int_{0}^{\infty}Y^{n}e^{-Y^{n}/2}dY \right]^{n}$$ $$=\frac{1}{2^{n}}\left[\int_{0}^{\infty}z^{1/n}e^{-z/2}dz\right]^{n}$$ with $z=Y^{n}$. This integral can be easily calculated\footnote{\label{2}\textit{Table of integrals, series and products},  I.S. Gradshteyn/I.M. Ryzhik, FI 2 779.} using the GAMMA function we have : $$ \langle \rho \rangle= \frac{2}{n^{n}}\left[\Gamma(\frac{1}{n})\right]^{n}$$
This result makes sense : we can calculate the exact limit as n $\rightarrow \infty$ : $\lim\limits_{n\to \infty}\langle \rho \rangle=\rho$ as illustrated on the figure below, which represent the expected value of the geometric sample mean versus n :

\vspace*{0.2cm}
\includegraphics[scale=0.19]{geosamplemean.png}

Therefore, assuming that expected value of the sample median can is \footnote{FINDCHIRP : an algorithm for detection of gravitational waves from inspiraling compact binaries, Appendix B.}: $$\langle x_{med}\rangle = 2 \times \sum_{l=1}^{n}\frac{(-1)^{l+1}}{l}$$
we have the following relation : $$\langle x_{med}\rangle\langle \rho\rangle=\frac{4}{n^{n}}\left[\Gamma(\frac{1}{n})\right]^{n}\sum_{l=1}^{n}\frac{(-1)^{l+1}}{l}$$
Thus the bias factor is : $$ \beta=\frac{2\sum_{l=1}^{n}\frac{(-1)^{l+1}}{l}}{\frac{2}{n^{n}}\left[\Gamma(\frac{1}{n})\right]^{n}}$$


\end{multicols}



\end{document}
