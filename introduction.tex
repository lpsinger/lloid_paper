\section{Introduction}
\label{sec:introduction}

% Outline:
%  - Low-mass inspirals are the most promising source of gravitational radiation for advanced \textsc{ligo}
% - Motivation for high throughput
%   - Point out challenges of advanced detectors: more templates, longer templates
% - Motivation for low latency
%   - Observational applications of near-real-time detection
%   - Mention S6 and VSR2, online running, EM followup
% - Layout of paper: description of method, procedure followed 

The coalescence of compact binary systems consisting of neutron stars
 (\textsc{ns}) and/or black holes (\textsc{bh}) is presently the best
understood and most actively pursued target source of gravitational
radiation for ground-based gravitational-wave (\GW{}) detectors.
As a compact binary system loses energy to \GW{}s, its
orbital separation decays, leading to a runaway inspiral with the \GW{}
amplitude and frequency increasing until the system eventually merges.  It is
thought that if a \textsc{ns} is involved, it may become tidally disrupted near
the merger and fuel a bright electromagnetic (\EM{})
counterpart~\citep{shibata:2007}. Observation of an \EM{} counterpart to a
candidate \GW{} event will vastly boost confidence in the \GW{} detection and
will often provide sufficient sky localization to determine
the source's host galaxy and therefore measure the source's redshift. The
\GW{} observation of this event will also provide a luminosity
distance measurement. With both quantities, we can
immediately measure the Hubble constant with unprecedented
accuracy and systematics independent of the standard cosmic distance
ladder~\citep{2010ApJ...725..496N}. The most efficient means of obtaining
such joint observations is through the rapid identification and dissemination
of candidate \GW{} events.

In the era of first-generation detectors, the \GW{} community initiated a
project to send alerts when potential \GW{} transients were observed in
order to trigger followup observations by \EM{} telescopes.  The best
demonstrated latencies were on the order of 1 hour%
~\citep{HugheyGWPAW2011}, which was an important
achievement, but far too late to catch prompt \EM{}
emission. In order to extract the maximum science from Advanced \LIGO{}
\CBC{} events, we have the ambition of reporting \GW{} candidates not minutes
\emph{after} the merger, as has already been accomplished, but seconds
\emph{before}.

Though compact binary coalescence (\CBC{}) events may be responsible for several
classes of \EM{} transients, \CBC{}s are in particular believed to be a
mechanism for short gamma-ray bursts (short \GRB{}s)~\citep{Lee:2005, nakar07}.
In this scenario, prompt \EM{} emission arises as shells of
relativistically out-flowing matter collide in the inner shock.
The same inner shocks, or potentially reverse shocks, can produce a bright
accompanying optical flash~\citep{Sari99}. Prompt emission is a probe into
the extreme initial conditions of the outflow, in contrast with afterglows,
which arise in the external shock with the local medium and are relatively
insensitive to initial conditions.

Optical flashes have only been observed for a handful of long
\GRB{}s~\citep{2011CRPhy..12..255A} by telescopes with extremely rapid response
or, in the case of \textsc{grb 080319b}, by pure serendipity, where several
telescopes were observing a previous \GRB{} in the same
field~\citep{2008Natur.455..183R}. The observed optical flashes peaked
within tens of seconds and decayed quickly.
Short \GRB{}s, on the other hand, typically
fade too quickly to even catch the tails of the afterglows in any band. Rapid
\GW{} transient alerts could enable the first observation of
prompt optical flashes and the rise of afterglows from short \GRB{}s, and, for
an exceptional event, perhaps even a glimpse of a
precursor~\citep{0004-637X-723-2-1711}.

We show that by examining the signal to noise ratio (\SNR{}) stream for
threshold crossings before the \GW{} signal leaves the detection band, it is
possible to trade some \SNR{} and sky localization
accuracy for negative latency.  Figure~\ref{fig:earlywarning} shows
projected early trigger rates for \textsc{ns}--\textsc{ns} binaries in Advanced
\LIGO\ assuming event rates predicted in \citet{Abadie:2010p10836} and anticipated
detector sensitivity described in \citet{ALIGONoise}.  The
most likely estimates indicate that $\sim$10 sources will produce an \SNR\
above 8 in the detector 10 seconds or more prior to the merger during one
year of observation, though the uncertainty spans orders of magnitude.
The gray bands give the 5 to 95\% confidence interval as described
in \citet{Abadie:2010p10836}.
%
\begin{figure}[h]
\plotone{figures/snr_in_time.pdf}
\caption{\label{fig:earlywarning} Expected number of \textsc{ns}--\textsc{ns}
sources that will be detectable $t$ seconds before coalescence.  The heavy
solid line is the most likely yearly rate estimate $\dot N_{\mathrm{re}}$
during Advanced \LIGO.  The shaded region represents the confidence interval
$[\dot N_{\mathrm{low}}, \dot N_{\mathrm{high}}]$ described in
\citep{Abadie:2010p10836}.  With an advanced detector in the `zero detuning,
high power' configuration \citep{ALIGONoise} and an \SNR\ threshold of 8,
we will observe a total of $\dot N_{\mathrm{re}} = 40$ events~yr$^{-1}$, but
$\sim$10~yr$^{-1}$ will be detectable within 10~s of merger and
$\sim$1~yr$^{-1}$ within 100~s.}
\end{figure}

In October 2010, \LIGO{} completed its sixth science run
(S6) and Virgo completed its third science run (VSR3).  While both
\LIGO{} detectors and Virgo were operating, several all-sky detection
pipelines operated in a low-latency configuration to send astronomical alerts,
namely \textsc{mbta}, Coherent WaveBurst, and
Omega~\citep{HugheyGWPAW2011, S6lowlatency}. The S6 analyses
achieved latencies of 30--60 minutes, which were dominated by a human vetting
process. Candidates were sent for \EM{} followup to several
telescopes; Swift, \textsc{rotse}, \textsc{tarot}, and Zadko~\citep{kanner2008,
HugheyGWPAW2011} took images of likely sky locations.  \textsc{mbta} achieved
the best \GW{} trigger-generation latencies of 2--5 minutes.  We
assume that in the advanced detector era the vetting process will be automated,
so current \GW{} search methodology and telescope actuation would
dominate latency.

Realizing advance detection of compact binary coalescences (\CBC{}s) will
require striking a balance between latency and throughput. \CBC{} searches
consist of banks of matched filters, or
cross-correlations between the data stream and a bank of nominal ``template''
signals.  There are many different implementations of matched filters, but most
have high throughput at the cost of high latency, or low latency at the cost of
low throughput.  The former are epitomized by the overlap-save algorithm~%
\citep{numerical-recipes-chapter-13} for frequency domain (\FD) convolution,
currently the preferred method in \GW{}
searches.  The most obvious example of the latter is the time domain
(\TD) convolution, which has no algorithmic latency.  However, its
computational complexity is quadratic in the length of the templates, so it is
prohibitively expensive for long templates.

Fortunately, the morphology of inspiral signals can be exploited to offset some
of the computational complexity of low-latency algorithms.  First, the signals
evolve slowly in frequency, so that they can be broken into contiguous
band-limited time intervals and processed at possibly lower sample rates.
Second, inspiral filter banks consist of highly similar templates, admitting
methods such as singular value decomposition (\SVD{}) to reduce the number of
templates \citep{Cannon:2010p10398}. We will use both aspects to demonstrate that a very
low latency analysis with advance detection of compact binary sources is
possible with current computing resources.  Assuming other technical sources of
latency can be reduced significantly, this should allow the possibility for
prompt alerts to be sent to the astronomical community.

The paper is organized as follows. First we provide an overview of our method
for detecting compact binary coalescence signals in an \earlywarning\ analysis.
We then describe the pipeline we have constructed that implements our method.
To validate the approach we present results of simulations and conclude with
some remarks on what remains to prepare for the advanced detector era.

