\section{Introduction}
\label{sec:introduction}

As a compact binary system loses energy to \GW{}s, its
orbital separation decays, leading to a runaway inspiral with the \GW{}
amplitude and frequency increasing until the system eventually merges.  It is
thought that if a \textsc{ns} is involved, it may become tidally disrupted near
the merger and fuel a bright electromagnetic (\EM{})
counterpart \citep{shibata:2007}. One way of obtaining such joint observations
is through the rapid identification and dissemination of candidate \GW{} events
for \EM{} followup.

In the first generation of ground-based laser interferometers, the \GW{}
community initiated a
project to send alerts when potential \GW{} transients were observed in order
to trigger followup observations by \EM{} telescopes.  The typical latencies
were 30--60 minutes \citep{HugheyGWPAW2011}, which was an important
achievement, but far too late to catch prompt \EM{} emission. We have the
ambition of reporting \GW{} candidates not minutes \emph{after} the merger, as
has already been accomplished, but seconds \emph{before}.

\subsection{\EM{} counterparts}

Though compact binary coalescence (\CBC{}) events may be responsible for several
classes of \EM{} transients, \CBC{}s are in particular believed to be a
mechanism for short gamma-ray bursts (short \GRB{}s) \citep{Lee:2005, nakar07}.
In this scenario, prompt \EM{} emission arises as shells of
relativistically out-flowing matter collide in the inner shock.
The same inner shocks, or potentially reverse shocks, can produce a bright
accompanying optical flash \citep{Sari99}. Prompt emission is a probe into
the extreme initial conditions of the outflow, in contrast with afterglows,
which arise in the external shock with the local medium and are relatively
insensitive to initial conditions.

Optical flashes have only been observed for a handful of long
\GRB{}s \citep{2011CRPhy..12..255A} by telescopes with extremely rapid response
or, in the case of \textsc{grb 080319b}, by pure serendipity, where several
telescopes were already observing the afterglow of another \GRB{} in the same
field of view~\citep{2008Natur.455..183R}. The observed optical flashes peaked
within tens of seconds and decayed quickly.
Short \GRB{}s, on the other hand, typically
fade too quickly to even catch the tails of the afterglows in any band. Rapid
\GW{} transient alerts could enable the first observation of
prompt optical flashes and the rise of afterglows from short \GRB{}s, and, for
a truly exceptional event, perhaps even a glimpse of a
precursor \citep{0004-637X-723-2-1711}.

\subsection{Early-warning \EM{} followup}

In October 2010, \LIGO{}\footnote{\url{http://www.ligo.org/}} completed its sixth science run
(S6) and Virgo\footnote{\url{http://www.ego-gw.it/}} completed its third science run (VSR3).
While both \LIGO{} detectors and Virgo were operating, several all-sky detection
pipelines operated in a low-latency configuration to send astronomical alerts,
namely \textsc{mbta}, Coherent WaveBurst, and
Omega \citep{HugheyGWPAW2011, S6lowlatency}. The S6 analyses
achieved latencies of 30--60 minutes, which were dominated by a human vetting
process. Candidates were sent for \EM{} followup to several
telescopes; Swift, \textsc{rotse}, \textsc{tarot}, and Zadko \citep{kanner2008,
HugheyGWPAW2011} imaged some of the most likely sky locations.  \textsc{mbta} achieved
the best \GW{} trigger-generation latencies of 2--5 minutes.  We
assume that in the advanced detector era the vetting process will be automated,
so current \GW{} search methodology and telescope actuation would
dominate latency.

\begin{figure}[h]
\plotone{figures/snr_in_time}
\caption{\label{fig:earlywarning}Expected number of \textsc{ns}--\textsc{ns}
sources that will be detectable by Advanced \LIGO\ a given number of seconds before coalescence.
The heavy solid line is the most realistic yearly rate estimate.  The shaded
region represents the 5 to 95\% confidence interval arising from uncertainty
in predicted event rates \citep{Abadie:2010p10836}.}
\end{figure}
%
It has been observed that by examining the signal to noise ratio (\SNR{}) stream for
threshold crossings before the \GW{} signal leaves the detection band, it is
possible to trade some \SNR{} and sky localization accuracy for pre-merger
detection. Figure~\ref{fig:earlywarning} shows projected early detection rates
for \textsc{ns}--\textsc{ns} binaries in Advanced \LIGO\ assuming event rates
predicted in \citet{Abadie:2010p10836} and anticipated detector sensitivity for
the `zero detuning, high power' configuration described in \citet{ALIGONoise}.
The most realistic estimates indicate that at an \SNR\ threshold of 8, we will
observe a total of 40~events~yr$^{-1}$, but $\sim$10~yr$^{-1}$ will be
detectable within 10~s of merger and $\sim$5~yr$^{-1}$ will be detectable
within 25~s of merger. These rates are shown with their substantial uncertainties in Figure~\ref{fig:earlywarning}.

\begin{table}[h]
\caption{\label{table:sky-localization-accuracy}Horizon distance, \SNR\ at merger, and area of 90\% confidence at selected times before merger for sources with expected detection rates of 40, 10, 1, and 0.1~yr$^{-1}$.}
\begin{center}
\begin{tabular}{rrrrrrr}
\tableline\tableline
rate & horiz. & final & \multicolumn{4}{c}{$A$(90\%) (deg$^2$)} \\
\cline{4-7}
yr$^{-1}$ & (Mpc) & \SNR\ & 25 s & 10 s & 1 s & 0 s \\
\tableline
40\phd\phn & 445 & 8.0 & 15500 & 3100 & 200 & 9.6 \\
10\phd\phn & 280 & 12.7 & 6100 & 1200 & 78 & 3.8 \\
1\phd\phn & 130 & 27.4 & 1300 & 260 & 17 & 0.8 \\
0.1 & 60 & 58.9 & 280 & 56 & 3.6 & 0.2 \\
\tableline
\end{tabular}
\end{center}
\end{table}
%
\begin{figure}[h]
\plotone{figures/loc_in_time}
\caption{\label{fig:sky-localization-accuracy}Area of the 90\% confidence
region as a function of time for a sources with anticipated detection rates
of 40, 10, 1, and 0.1~yr$^{-1}$. The heavy dot indicates the
time at which the accumulated \SNR\ exceeds a threshold of 8.}
\end{figure}

\EM{} followup requires estimating the location of the \GW{} source. The
localization uncertainty is dominated by uncertainty in the time of arrival of
the \GW{}s, which is determined by the signal's effective bandwidth and \SNR{}
\citep{Fairhurst2009}. Table~\ref{fig:earlywarning} and Figure~\ref{fig:sky-localization-accuracy} show the estimated
minimum 90\% confidence area versus time of the loudest
coalescence events detectable by Advanced \LIGO{} and Advanced Virgo.  Once per year,
we expect to observe an event with a final \SNR\ of 27.4 whose location can be
constrained to about 1300~deg$^2$ (3.1\% of the sky) 25~s before merger, 260~deg$^2$
(0.63\% of the sky) 10~s before merger, and 0.82~deg$^2$ (0.0020\% of the sky) at merger.

It is infeasible to search hundreds of square degrees for a prompt
counterpart. For
reference, \textsc{lsst} should see first light around the same time as
Advanced \LIGO{} and will have an unparalleled field of view of 9.6 deg$^2$
\citep{2008arXiv0805.2366I}.  However, if we can take galactic luminosities
to accurately reflect mass and thus be a proxy for the \CBC{} event rate, it is
possible to reduce the localization uncertainty by only looking at galaxies from a
catalog that lie near the sky location and luminosity distance estimate from the \GW{}
signal~\citep{galaxy-catalog} as was done in S6/VSR3.
Within the expected Advanced \LIGO{} \textsc{ns}--\textsc{ns} horizon
distance, the number of galaxies that can produce a given signal
amplitude is much larger than the number for initial \LIGO{} and thus the
catalog will not be as useful for downselecting pointings generally. However,
our exceptional sources will necessarily be extremely nearby. Within this
reduced volume there will be fewer galaxies to consider for a given
candidate and catalog completeness will be less of a concern.
This should allow us to shrink the 90\% confidence area substantially.

\subsection{Latency in \CBC{} search algorithms}

Realizing advance detection of \CBC{}s will
require striking a balance between latency and throughput. \CBC{} searches
consist of banks of matched filters, or
cross-correlations between the data stream and a bank of nominal ``template''
signals.  There are many different implementations of matched filters, but most
have high throughput at the cost of high latency, or low latency at the cost of
low throughput.  The former are epitomized by the overlap-save algorithm~%
\citep{numerical-recipes-chapter-13} for frequency domain (\FD) convolution,
currently the preferred method in \GW{}
searches.  The most obvious example of the latter is the time domain
(\TD) convolution, which is latency-free.  However, its
computational complexity is quadratic in the length of the templates, so it is
prohibitively expensive for long templates.

Fortunately, the morphology of inspiral signals can be exploited to offset some
of the computational complexity of low-latency algorithms.  First, the signals
evolve slowly in frequency, so that they can be broken into contiguous
band-limited time intervals and processed at possibly lower sample rates.
Second, inspiral filter banks consist of highly similar templates, admitting
methods such as singular value decomposition (\SVD{}) to reduce the number of
templates \citep{Cannon:2010p10398}. We will use both aspects to demonstrate that a very
low latency analysis with advance detection of compact binary sources is
possible with current computing resources.  Assuming other technical sources of
latency can be reduced significantly, this should allow the possibility for
prompt alerts to be sent to the astronomical community.

The paper is organized as follows. First we provide an overview of our method
for detecting compact binary coalescence signals in an \earlywarning\ analysis.
We then describe the pipeline we have constructed that implements our method.
To validate the approach we present results of simulations and conclude with
some remarks on what remains to prepare for the advanced detector era.

