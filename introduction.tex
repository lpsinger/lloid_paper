\section{Introduction}
\label{sec:introduction}

% I would like to say:
%  - Low mass inspirals are the most promising source of gravitational radiation for advanced LIGO
% - Mention S6 and VSR2, online running, EM followup
% - Observational applications of near-realtime detection
% - Point out challenges of advanced detectors: more templates, longer templates
% - Advanced LIGO timetable
% - Einstein telescope
% - Layout of paper: description of method, procedure followed 


\begin{comment}

However, the computational requirements of conventional detection strategies --- especially those that are well suited to low latency applications --- will be greatly increased by this upgrade for two reasons.  First, advanced detectors will have much improved low-frequency sensitivity, so chirp signals from coalescing binaries will remain in the detectors' band for longer.  This will demand that matched filter based searches use longer templates, requiring more memory and more cycles to process.  Second, background noise for advanced detectors will be more spectrally uniform, so it will be possible to resolve the intrinsic parameters of a source that determine the time-frequency structure of the gravitational wave signal with much greater accuracy.  This will require that matched filter bank based searches employ more matched filters, again demanding more memory and more cycles.

\end{comment}

\begin{comment}

\editorial{This paragraph has too many citations.} Modern gravitational-wave observatories are likely to deliver on the promise 
that their name implies by observing multiple gravitational-wave signals from
compact binary coalescence within the coming decade \cite{kopparapu2008}. 
There is now a world-wide network of gravitational-wave observatories 
including the laser interferometers
\textsc{ligo}, Virgo, \textsc{geo}, and \textsc{tama}, which should assure that coincident detections 
will be possible among at least three separate sites lending confidence
to any detections~\cite{Abbott:2005qm, beauville2008, beauville2006, blackburn2005, LIGOS1instpaper}.  \editorial{This is a lot of citations just to say that the 4 detectors exist.  Maybe if they were scattered throughout the paragraph?}
However, having an electromagnetic or neutrino counterpart to a 
gravitational-wave detection would not only increase the confidence in 
the detection but would also greatly improve the astrophysical information
available from the event.

\end{comment}

In a series of detector upgrades that are now under way, \textsc{ligo} and Virgo will become the first advanced gravitational wave detectors, gaining a tenfold improvement in amplitude sensitivity \citeneeded{} and a corresponding thousandfold increase in observable volume in the local universe.  It is anticipated that these detectors will be able to sense as many as 40 events per year \cite{Abadie:2010p10836}.

As the gravitational wave detection horizon pushes outward, the ability to
detect signals in near realtime will become valuable.  Having an
electromagnetic or neutrino counterpart to a gravitational-wave detection would
not only increase the confidence in the detection but will also greatly improve
the astrophysical information available from the event.  Most models that
predict simultaneous gravitational-wave and electromagnetic observations also
predict that the peak amplitude of electromagetic radiation will occur soon
after gravitational-wave emission~\cite{sylvestre2003}.  Thus in order to
maximize the chance of a successful electromagnetic followup the latency of
gravitational-wave signal analysis must be made minimal.

\editorial{This is old news, and doesn't seem relevant to me: EM-directed
gravitational wave searches don't benefit from low-latency detection
pipelines.; Drew: I agree} Work has already commenced to trigger
gravitational-wave searches from electromagnetic
observations~\cite{triggeredsearches2008}. \textsc{ligo} observations ruled out
the possibility of GRB070201 originating from the merger of a \textsc{nsns} or
\textsc{nsbh} system in the Andromeda galaxy~\cite{GRB070201}.

In \oldstylenums{2010}, \textsc{ligo} and Virgo completed a period of joint data taking during which several all-sky detection pipelines operated in a low-latency configuration \citeneeded.  Promising detection candidates were promptly sent for electromagnetic followup \citeneeded{} to several telescopes including Swift, \textsc{rotse}, \textsc{tarot}, and Zadko.


This question has been investigated and it seems  at least possible that
gravitational-wave detections could provide a region of the sky to prompt
electromagnetic observation followups~\cite{sylvestre2003}.  \editorial{Cite an
in-preparation paper of Larry's? Drew: When it is on arXiv, yes.} Much work is
currently underway in providing the best possible source localization from
gravitational-wave detector networks~\cite{markowitz2008, raymond2008,
cavalier2006}.  Collaborations are currently forming to provide infrastructure
for the gravitational-wave Astronomers to provide targets of opportunity for
electromagnetic astronomers~\cite{kanner2008}.  Most models that predict
coincident gravitational-wave and electromagnetic observations also predict
that the peak electromagetic fluence will occur soon after gravitational-wave
emission~\cite{sylvestre2003}.  Thus in order to maximize the chance of a
successful electromagnetic followup the latency of gravitational-wave signal
analysis must be made minimal.

The parameter space of compact binary coalescence signals is large~\cite{Owen:1995tm, Owen:1998dk}.  It is a
computationally burdensome task to analyze these signals with even moderate
latency~\cite{Abbott:2007xi}.  
This work will describe how to exploit degeneracy in the signal
parameter space to answer more quickly whether or not a gravitational-wave 
is present. Specifically we will explore using the singular value decomposition (\textsc{svd}) to reduce the
effective number of filters required to search the data.  We note that others
have applied the use of \textsc{svd} to gravitational wave data analysis to analyze
optimal gravitational-wave burst detection~\cite{bradyraymajumder2004, heng2008} and coherent networks 
of detectors~\cite{wen2008} .

The paper is organized as follows.  First we provide an overview of the 
standard method for detecting compact binary coalescence signals and
describe how it can be modified to accomodate low latency analysis.  We
then describe the pipeline we have constructed to implement these changes.
To validate the approach we present results of simulations and finish
with some concluding remarks. 
