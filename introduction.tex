\section{Introduction}
\label{sec:introduction}

As a compact binary system loses energy to gravitational waves (\GW{}s), its
orbital separation decays, leading to a runaway inspiral with the \GW{}
amplitude and frequency increasing until the system eventually merges.
If a neutron star (\textsc{ns}) is involved, it may become tidally disrupted near
the merger and fuel a bright electromagnetic (\EM{}) counterpart
\citep{shibata:2007}.  The \GW{} signal is in principle detectable
before merger.  Effort from both the \GW{} and astronomy communities may make
it possible to use \GW{} observations as an early warning trigger for \EM{}
followup. In the first generation of ground-based laser interferometers, the \GW{}
community initiated a project to send alerts when potential \GW{} transients
were observed in order to trigger followup observations by \EM{} telescopes.
The typical latencies were 30 minutes \citep{HugheyGWPAW2011}, which was an
important achievement, but too late to catch any prompt \EM{}
emission. We have the ambition of reporting \GW{} candidates not minutes after
the merger, as has already been accomplished, but seconds before.  We present
one essential ingredient in this effort, a computationally inexpensive
low-latency filtering algorithm for detecting inspiral signals in \GW{}
data.  We also discuss the prospects for advanced \GW\ detectors and
discuss other areas of work that would be required for rapid analysis.

\subsection{\EM\ counterparts}

Compact binary coalescence (\CBC) events are thought to be a
mechanism for short gamma-ray bursts (short \GRB{}s) \citep{Lee:2005, nakar07}.
In this scenario, prompt gamma-ray emission arises on the accretion timescale
of a central compact object formed after the merger, $0.1$--$1.0$\,s
\citep{Janka1999}.  Interestingly, roughly a quarter to half of observed short
\GRB{}s also exhibit extended emission of $30$--$100$\,s in duration beginning
$\sim$$10$\,s after the \GRB{} and carrying comparable fluence to the initial
outburst.  Potential explanations for the emission are delayed fall-back
accretion caused by $r$-process heating \citep{Metzger2010} or the formation of
a proto-magnetar that interacts with
ejecta \citep{Bucciantini2011}.  A proto-magnetar may
form from accretion-induced collapse or the merger of two \textsc{ns}s.  An
accompanying \GW{} observation would confirm the latter.

Optical flashes have only been observed for a handful of long \GRB{}s
\citep{2011CRPhy..12..255A} by telescopes with extremely rapid response or, in
the case of \textsc{grb 080319b}, by pure serendipity, where several telescopes
were already observing the afterglow of another \GRB{} in the same field of
view \citep{2008Natur.455..183R}. The observed optical flashes peaked within
tens of seconds and decayed quickly.  Short \GRB{}s, on the other hand,
typically fade so quickly that it is difficult to catch even the tails of the afterglows in any
band. Rapid \GW{} transient alerts could enable the first observation of prompt
optical flashes and the rise of afterglows from short \GRB{}s.

\subsection{Early-warning \EM{} followup}

In October 2010, \LIGO{}\footnote{\url{http://www.ligo.org/}} completed its
sixth science run (S6) and Virgo\footnote{\url{http://www.ego-gw.it/}}
completed its third science run (VSR3).  While both \LIGO{} detectors and Virgo
were operating, several all-sky detection pipelines operated in a low-latency
configuration to send astronomical alerts, namely \textsc{mbta}, Coherent
WaveBurst, and Omega \citep{HugheyGWPAW2011, S6lowlatency}. The S6 analyses
achieved latencies of 30--60 minutes, which were dominated by a human vetting
process. Candidates were sent for \EM{} followup to several telescopes; Swift,
\textsc{rotse}, \textsc{tarot}, and Zadko \citep{kanner2008, HugheyGWPAW2011}
imaged some of the most likely sky locations.  \textsc{mbta} achieved the best
\GW{} trigger-generation latencies of 2--5 minutes.

\begin{figure}[h]
\plotone{figures/snr_in_time}
\caption{\label{fig:earlywarning}Expected number of \textsc{ns}--\textsc{ns}
sources that could be detectable by Advanced \LIGO\ a given number of seconds
before coalescence.  The heavy solid line is the most realistic yearly rate
estimate.  The shaded region represents the 5 to 95\% confidence interval
arising from uncertainty in predicted event rates \citep{Abadie:2010p10836}.}
\end{figure}
%
Examining the signal to noise ratio (\SNR{}) stream for threshold crossings
before the \GW{} signal leaves the detection band makes it possible to trade
some \SNR{} and sky localization accuracy for pre-merger detection.
Figure~\ref{fig:earlywarning} shows projected early detectability rates for
\textsc{ns}--\textsc{ns} binaries in Advanced \LIGO\ assuming event rates
predicted in \citet{Abadie:2010p10836} and anticipated detector sensitivity for
the `zero detuning, high power' configuration described in \citet{ALIGONoise}.
The most realistic estimates indicate that at an \SNR\ threshold of 8, we will
observe a total of 40\,events\,yr$^{-1}$. $\sim$10\,yr$^{-1}$ will be detectable
within 10\,s of merger and $\sim$5\,yr$^{-1}$ will be detectable within 25\,s of
merger if analysis can proceed with near zero latency. These rates are shown
with their substantial uncertainties in Figure~\ref{fig:earlywarning}.

We emphasize that Figure~\ref{fig:earlywarning} provides only the lower bound
for early detection and should serve as a baseline to understand the potential
of early warning \GW{} searches in a ideal world.  Real analysis will have
sources of latency that are unavoidable meaning that the actual detection will
occur with some delay. 


\begin{table}[h]
\caption{\label{table:sky-localization-accuracy}Horizon distance, \SNR\ at
merger, and area of 90\% confidence at selected times before merger for sources
with expected detectability rates of 40, 10, 1, and 0.1\,yr$^{-1}$.}
\begin{center}
\begin{tabular}{rrrrrrr}
\tableline\tableline
rate & horiz. & final & \multicolumn{4}{c}{$A$(90\%) (deg$^2$)} \\
\cline{4-7}
yr$^{-1}$ & (Mpc) & \SNR\ & 25 s & 10 s & 1 s & 0 s \\
\tableline
40\phd\phn & 445 & 8.0 & 15500 & 3100 & 200 & 9.6 \\
10\phd\phn & 280 & 12.7 & 6100 & 1200 & 78 & 3.8 \\
1\phd\phn & 130 & 27.4 & 1300 & 260 & 17 & 0.8 \\
0.1 & 60 & 58.9 & 280 & 56 & 3.6 & 0.2 \\
\tableline
\end{tabular}
\end{center}
\end{table}
%
\begin{figure}[h]
\plotone{figures/loc_in_time}
\caption{\label{fig:sky-localization-accuracy}Area of the 90\% confidence
region as a function of time before coalescence for sources with anticipated
detectability rates of 40, 10, 1, and 0.1\,yr$^{-1}$. The heavy dot indicates
the time at which the accumulated \SNR\ exceeds a threshold of 8.}
\end{figure}

Before discussing realistic latencies, it is important to note that \EM{}
followup requires estimating the location of the \GW{} source. The localization
uncertainty can be estimated from the uncertainty in the time of arrival of the \GW{}s,
which is determined by the signal's effective bandwidth and \SNR{}
\citep{Fairhurst2009}, though this overestimates the area on the sky for large
timing errors.  Table~\ref{fig:earlywarning} and
Figure~\ref{fig:sky-localization-accuracy} show the estimated minimum 90\%
confidence area versus time of the loudest coalescence events detectable by
Advanced \LIGO{} and Advanced Virgo.  Once per year, we expect to observe an
event with a final \SNR\ of $\approx 27$ whose location can be constrained to about
1300\,deg$^2$ (3.1\% of the sky) within 25\,s of merger,
260\,deg$^2$ (0.63\% of the sky) within 10\,s of merger, and
0.82\,deg$^2$ (0.0020\% of the sky) at merger.

It is infeasible to search hundreds of square degrees for a prompt counterpart.
For reference, \textsc{lsst} should see first light around the same time as
Advanced \LIGO{} and will have an unparalleled field of view of 9.6 deg$^2$
\citep{2008arXiv0805.2366I}.  However, 
it is possible to reduce the localization uncertainty by only looking at
galaxies from a catalog that lie near the sky location and luminosity distance
estimate from the \GW{} signal~\citep{galaxy-catalog} as was done in S6/VSR3.
Within the expected Advanced \LIGO{} \textsc{ns}--\textsc{ns} horizon distance,
the number of galaxies that can produce a given signal amplitude is much larger
than in Initial \LIGO{} and thus the catalog will not be as useful
for downselecting pointings generally. However, exceptional \GW{} sources will
necessarily be extremely nearby. Within this reduced volume there will be fewer
galaxies to consider for a given candidate and catalog completeness will be
less of a concern.  This should reduce the 90\% confidence area substantially.
Also future \GW{} detectors, should they be built, will improve the \SNR\ and
likewise the error box.

\subsection{Latency in \CBC{} searches}

There were a number of sources of latency associated with \CBC{}
analysis in S6/VSR3 \citep{HugheyGWPAW2011}, listed here.
%
\begin{itemize}
%
\item {\it Data acquisition and aggregation ($\gtrsim$100~ms):}
The \LIGO\ data acquisition system collects data from detector subsystems 16
times a second~\citep{Bork2001}. Data are also copied from all of the \GW\
observatories to the analysis clusters over the Internet, which is capable of
high bandwidth but perhaps only modest latency.  Together, these introduce a
latency of $\gtrsim 100$~ms.  These technical sources of latency could be reduced
with significant engineering and capital investments, but they are minor compared
to any of the other sources of latency.
%
\item {\it Data conditioning
($\sim$1~min):} Science data must be calibrated using the detector's frequency
response to gravitational radiation.  Currently, data are calibrated in blocks of
16~s.  Within $\sim$1~min, data quality is assessed in order to create veto flags.
These are both technical sources of latency that might be addressed with improved
calibration and data quality software for advanced detectors.
%
\item {\it Trigger generation (2--5~min):} Low-latency data analysis pipelines
deployed in S6/VSR3 achieved an impressive latency of minutes.  However, second
to the human vetting process, this dominated the latency of the entire \EM\
followup process.  Even if no other sources of latency existed, this trigger
generation latency is too long to catch prompt or even extended emission.
Low-latency trigger generation will become more challenging with advanced detectors
because inspiral signals will stay in band up to ten times longer.  In this work,
we will focus on reducing this source of latency.
%
\item {\it Alert generation
(2--3~min):} S6/VSR3 saw the introduction of low-latency astronomical
alerts, which required gathering event parameters and sky localization from the
various online analyses, downselecting the events, and calculating pointings.
If other sources of latency improve, the technical latency associated with this
infrastructure could dominate, so work should be done to improve it.
%
\item {\it Human validation (10--20~min):} Because the new alert system was
commissioned during S6/VSR3, humans examined each candidate for obvious flaws
before publishing an alert. This was by far the largest latency during S6/VSR3.
Hopefully, confidence in the system will grow to the point where no human intervention
is necessary before alerts are sent, so we give it no further consideration here.
%
\end{itemize}

This work will focus on reducing the latency of trigger production.  Realizing
advance detection of \CBC{}s will require striking a balance between latency
and throughput of filtering \GW\ data. \CBC{} searches consist of banks of
matched filters, or cross-correlations between the data stream and a bank of
nominal ``template'' signals.  There are many different implementations of
matched filters, but most have high throughput at the cost of high latency, or
low latency at the cost of low throughput.  The former are epitomized by the
overlap-save algorithm
for frequency domain (\FD) convolution, currently the preferred method in \GW{}
searches.  The most obvious example of the latter is the time domain (\TD)
convolution, which is latency-free.  However, its computational complexity is
quadratic in the length of the templates, so it is prohibitively expensive for
long templates.

Fortunately, the morphology of inspiral signals can be exploited to offset some
of the computational complexity of low-latency algorithms.  First, the signals
evolve slowly in frequency, so that they can be broken into contiguous
band-limited time intervals and processed at possibly lower sample rates.
Second, inspiral filter banks consist of highly similar templates, admitting
methods such as singular value decomposition (\SVD{}) to reduce the number of
templates \citep{Cannon:2010p10398}. We will use both aspects to demonstrate
that a very low latency detection statistic is possible with current computing
resources.  Assuming the other technical sources of latency can be reduced
significantly, this should allow the possibility for prompt alerts to be sent
to the astronomical community.

The paper is organized as follows. First we provide an overview of our method
for detecting compact binary coalescence signals in an \earlywarning\ analysis.
We then describe the pipeline we have constructed that implements our method.
To validate the approach we present results of simulations and conclude with
some remarks on what remains to prepare for the advanced detector era.

