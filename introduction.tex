\section{Introduction}
\label{sec:introduction}

% Outline:
%  - Low-mass inspirals are the most promising source of gravitational radiation for advanced \textsc{ligo}
% - Motivation for high throughput
%   - Point out challenges of advanced detectors: more templates, longer templates
% - Motivation for low latency
%   - Observational applications of near-realtime detection
%   - Mention S6 and VSR2, online running, EM followup
% - Layout of paper: description of method, procedure followed 


Low-mass inspiraling binaries are the most promising sources of gravitational
radiation for Advanced \textsc{ligo}. Observation demonstrates that they must
exist in the local Universe in some abundance~\cite{Abadie:2010p10836}.
[science] In the
final stages of inspiral, the neutron star is tidally disrupted, providing the
matter required to produce bright electromagnetic
counterparts~\cite{shibata:2007}. Conveniently, the gravitational waveforms are
understood and implemented to the level required by Advanced \textsc{ligo} to
detect~\cite{BuonannoIyerOchsnerYiSathya2009}, allowing the sensitive technique
of matched filtering.

The parameter space of compact binary coalescence signals is
large~\cite{Owen:1995tm, Owen:1998dk} leading to large filter banks and
significant computational cost. Following the series of detector upgrades that
are now under way, \textsc{ligo} and Virgo will attain their Advanced
configurations, gaining a tenfold improvement in amplitude sensitivity over
Initial configurations~\cite{advLIGOrefdesign,advVirgorefdesign} and a
corresponding thousandfold increase in observable volume in the local Universe.
The computational requirements of conventional detection strategies will be
greatly increased by this upgrade for two reasons. First, advanced detectors
will have much improved low-frequency sensitivity, so chirp signals from
coalescing binaries will remain in the detectors' band for a significantly
longer time. This will demand that matched filter based searches use longer
templates, requiring more memory and more cycles to process. Second, background
noise for advanced detectors will be more spectrally uniform, so it will be
possible to resolve the intrinsic parameters of a source that determine the
time-frequency structure of the gravitational wave signal with much greater
accuracy. This will require that matched filter bank based searches employ
more matched filters, again demanding more memory and more cycles.

Beyond raw throughput, the ability to detect signals in near realtime will
become increasingly valuable as the gravitational-wave detection horizon pushes
outward.  Having an electromagnetic or neutrino counterpart to a
gravitational-wave detection would not only increase the confidence in the
detection but will also greatly increase the astrophysical information
available from the event. Most models that predict simultaneous
gravitational-wave and electromagnetic observations also predict that the peak
amplitude of electromagetic radiation will occur soon after gravitational-wave
emission~\cite{sylvestre2003}. Thus in order to maximize the chance of a
successful electromagnetic followup the latency of gravitational-wave signal
analysis must be made minimal.

In \oldstylenums{2010}, \textsc{ligo} and Virgo completed S6/VSR3,
a short period of
joint data taking during which several all-sky detection pipelines operated in
a low-latency configuration \citeneeded. A few candidates of moderate
significance were promptly sent for electromagnetic followup \citeneeded{} to
several telescopes including Swift, \textsc{rotse}, \textsc{tarot}, and Zadko.
They achieved latencies of BLAH, but required human vetting of each candidate
before alerting the telescope partners. As confidence in the infrastructure
increases, humans can be removed from the loop, setting a demonstrated previous
best latency at BLAH.

This work will describe how to exploit degeneracy in the signal parameter space
to answer more quickly and economically whether or not a gravitational wave is
present.
\editorial{nvf: I use some rather different words here to summarize the
techniques. I don't want to simply repeat words from later, but I don't want
to jar a reader with changing jargon either. Feel free to edit.}
Specifically, we will exploit the time-frequency structure of chirp
signals to downsample different parts of the waveform, use the singular value
decomposition (\textsc{svd}) factors to identify the redundancy of the bank and
reduce the effective number of filters required to search the data, and use
matched subspace filters built from \textsc{svd} to identify when computation
can be avoided.  We note that others have applied the use of \textsc{svd} to
gravitational wave data analysis to analyze optimal gravitational-wave burst
detection~\cite{bradyraymajumder2004, heng2008} and coherent networks of
detectors~\cite{wen2008}.

The paper is organized as follows.  First we provide an overview of the 
standard method for detecting compact binary coalescence signals and
describe how it can be modified to accomodate low latency analysis.  We
then describe the pipeline we have constructed to implement these changes.
To validate the approach we present results of simulations and finish
with some concluding remarks. 
