\section{Introduction}
\label{sec:introduction}

As a compact binary system loses energy to gravitational waves (\GW{}s), its
orbital separation decays, leading to a runaway inspiral with the \GW{}
amplitude and frequency increasing until the system eventually merges.  If a
neutron star (\NS{}) is involved, it may become tidally disrupted near
the merger and fuel an electromagnetic (\EM{}) counterpart
\citep{shibata:2007}.  Effort from both the \GW{} and astronomy communities may make it
possible to use \GW{} observations as an early warning trigger for \EM{}
followup. In the first generation of ground-based laser interferometers, the
\GW{} community initiated a project to send alerts when potential \GW{}
transients were observed in order to trigger followup observations by \EM{}
telescopes.  The typical latencies were 30 minutes \citep{HugheyGWPAW2011},
which was an important achievement, but too late to catch any prompt \EM{}
emission.  Since the \GW\ signal is in principle detectable even before the tidal
disruption, one may have the ambition of reporting \GW\ candidates not minutes
after the merger, but seconds before.  We explore one essential ingredient of this
problem, a computationally inexpensive low-latency filtering algorithm for detecting
inspiral signals in \GW{} data.  We also consider the prospects for advanced \GW{}
detectors and discuss other areas of work that would be required for rapid analysis.

Compact binary coalescence (\CBC{}) is a plausible progenitor for most short
gamma-ray bursts (short \GRB{}s) \citep{Lee:2005, nakar07}, but the
association is not iron-clad \citep{2011ApJ...727..109V}. The tidally
disrupted material falls onto the newly formed, rapidly spinning compact object
and is accelerated in jets along the spin axis with a timescale of
$0.1$--$1$~s after the merger \citep{Janka1999}, matching the short \GRB{}
duration distribution well. Prompt \EM{} emission including the \GRB{}
can arise as fast outflowing
matter collides with slower matter ejected earlier in inner shocks. The same
inner shocks, or potentially reverse shocks, can produce an accompanying
optical flash \citep{Sari99}. The prompt emission is a probe into the extreme
initial conditions of the outflow, in contrast with afterglows, which arise in
the external shock with the local medium and are relatively insensitive to
initial conditions. Optical flashes have been observed for a handful of long
\GRB{}s \citep{2011CRPhy..12..255A} by telescopes with extremely rapid response
or, in the case of GRB 080319b, by pure serendipity, where several
telescopes were already observing the afterglow of another \GRB{} in the same
field of view \citep{2008Natur.455..183R}. The observed optical flashes peaked
within tens of seconds and decayed quickly. For short \GRB{} energy balance and
plasma density,
however, the reverse shock model predicts a peak flux in radio, approximately
20 minutes after the \GRB{}, but also a relatively faint optical flash
\citep{nakar07}; for a once-per-year Advanced LIGO event at $130$\,Mpc, the
radio flux will peak around $9$\,GHz at $\sim$$5$\,mJy, with emission in the
$R$-band at $\sim$$19^\mathrm{th}$ magnitude. Interestingly, roughly a quarter to half of observed short
\GRB{}s also exhibit extended X-ray emission of $30$--$100$\,s in duration
beginning $\sim$$10$\,s after the \GRB{} and carrying comparable fluence to the
initial outburst. This can be explained if the merger results in the formation
of a proto-magnetar that interacts with ejecta \citep{Bucciantini2011}. Rapid
\GW{} alerts would enable joint \EM{} and \GW{} observations to confirm
the short \GRB{}-\CBC{} link and allow the early \EM{}
observation of exceptionally nearby and thus bright events.

In October 2010, \LIGO{}\footnote{\url{http://www.ligo.org/}} completed its
sixth science run (S6) and Virgo\footnote{\url{http://www.ego-gw.it/}}
completed its third science run (VSR3).  While both \LIGO{} detectors and Virgo
were operating, several all-sky detection pipelines operated in a low-latency
configuration to send astronomical alerts, namely \mbta{}, Coherent
WaveBurst, and Omega \citep{HugheyGWPAW2011, S6lowlatency2, S6lowlatency3, S6lowlatency4}.
\mbta{} achieved the best \GW{} trigger-generation latencies of 2--5 minutes.
Alerts were sent with latencies of 30--60 minutes, dominated by human vetting.
Candidates were sent for \EM{} followup to several telescopes; Swift,
LOFAR, ROTSE, TAROT, QUEST, SkyMapper,
Liverpool Telescope, Pi of the Sky, Zadko, and Palomar Transient Factory
\citep{kanner2008, HugheyGWPAW2011} imaged some of the most likely sky
locations.

There were a number of sources of latency associated with the search for
\CBC{} signals in S6/VSR3 \citep{HugheyGWPAW2011}, listed here.

\paragraph{Data acquisition and aggregation ($\gtrsim$100~ms)}%
The \LIGO\ data acquisition system collects data from detector subsystems 16
times a second~\citep{Bork2001}. Data are also copied from all of the \GW\
observatories to the analysis clusters over the Internet, which is capable of
high bandwidth but only modest latency.  Together, these introduce a
latency of $\gtrsim$$100$~ms.  These technical sources of latency could be reduced
with significant engineering and capital investments, but they are minor compared
to any of the other sources of latency.

\paragraph{Data conditioning ($\sim$1~min)}%
Science data must be calibrated using the detector's frequency
response to gravitational radiation.  Currently, data are calibrated in blocks of
16~s.  Within $\sim$1~min, data quality is assessed in order to create veto flags.
These are both technical sources of latency that might be addressed with improved
calibration and data quality software for advanced detectors.

\paragraph{Trigger generation (2--5~min)}%
Low-latency data analysis pipelines
deployed in S6/VSR3 achieved an impressive latency of minutes.  However, second
to the human vetting process, this dominated the latency of the entire \EM\
followup process.  Even if no other sources of latency existed, this trigger
generation latency is too long to catch prompt or even extended emission.
Low-latency trigger generation will become more challenging with advanced detectors
because inspiral signals will stay in band up to ten times longer.  In this work,
we will focus on reducing this source of latency.

\paragraph{Alert generation (2--3~min)}%
S6/VSR3 saw the introduction of low-latency astronomical
alerts, which required gathering event parameters and sky localization from the
various online analyses, downselecting the events, and calculating telescope pointings.
If other sources of latency improve, the technical latency associated with this
infrastructure could dominate, so work should be done to improve it.

\paragraph{Human validation (10--20~min)}%
Because the new alert system was commissioned during S6/VSR3, all alerts were subjected
to quality control checks by human operators before they were disseminated.
This was by far the largest source of latency during S6/VSR3.
Hopefully, confidence in the system will grow to the point where no human intervention
is necessary before alerts are sent, so we give it no further consideration here.

\paragraph{}

This work will focus on reducing the latency of trigger production.  Data analysis
strategies for advance detection of \CBC{}s will have to strike a balance between latency
and throughput. \CBC{} searches consist of banks of
matched filters, or cross-correlations between the data stream and a bank of
nominal ``template'' signals.  There are many different implementations of
matched filters, but most have high throughput at the cost of high latency, or
low latency at the cost of low throughput.  The former are epitomized by the
overlap-save algorithm for frequency domain (\FD) convolution, currently the
preferred method in \GW{} searches.  The most obvious example of the latter is
the time domain (\TD) convolution, which is latency-free.  However, its cost in floating
point operations per second is linear in the length of the templates, so it is
prohibitively expensive for long templates.  \citet{shaunIIR} explored decomposing
\CBC\ matched filters into banks of infinite impulse response (IIR)
filters as a means to reduce the computational cost of latency-free data analysis.  They
demonstrate a factor of 250 improvement in floating point operations per second
over the standard \TD\ method while faithfully reproducing the transfer functions of the
matched filters above 20~Hz.  We will demonstrate a significantly greater speedup.

Fortunately, the morphology of inspiral signals can be exploited to offset some
of the computational complexity of low-latency algorithms.  First, the signals
evolve slowly in frequency, so that they can be broken into contiguous
band-limited time intervals and processed at possibly lower sample rates.
Second, inspiral filter banks consist of highly similar templates, admitting
methods such as singular value decomposition (\SVD{}) to reduce the number of
templates \citep{Cannon:2010p10398}. We will use both aspects to demonstrate
that a very low latency detection statistic is possible with current computing
resources.  Assuming the other technical sources of latency can be reduced
significantly, this could make it possible to send prompt alerts to the
astronomical community.

The paper is organized as follows.  First, we discuss prospects for early-warning
detection.  Then, we provide an overview of our novel method for detecting compact
binary coalescence signals with extremely low latency. We then describe a prototype
implementation using open source signal processing software.  To validate our approach
we present results of simulations.  We conclude with some remarks on what remains to
prepare for the advanced detector era.
