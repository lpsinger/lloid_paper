\section{Introduction}
\label{sec:introduction}

% Outline:
%  - Low-mass inspirals are the most promising source of gravitational radiation for advanced \textsc{ligo}
% - Motivation for high throughput
%   - Point out challenges of advanced detectors: more templates, longer templates
% - Motivation for low latency
%   - Observational applications of near-realtime detection
%   - Mention S6 and VSR2, online running, EM followup
% - Layout of paper: description of method, procedure followed 


The coalescence of compact binary systems consisting of neutron stars and/or
black holes is the most promising source of gravitational radiation for
ground-based gravitational-wave detectors such as Advanced
\textsc{ligo}~\cite{ALIGOWeb}, Virgo~\cite{AVirgoWeb}, GEO~\cite{GEOWeb}, and
LCGT~\cite{LCGTWeb}.
A detection would strongly constrain mass and
spin~\cite{finn1993,Poisson:1995ef}, and with sufficient signal strength, the
neutron star equation of state~\cite{flanagan:021502,Read:2009}, general
relativity in the strong-field regime~\cite{Will:2005va}, the graviton
mass~\cite{PhysRevD.80.044002,Keppel:2010qu}, and the Lorentz Invariance
principle~\cite{Ellis2006402}.
Radio observation of binary pulsar systems in the Galaxy show the existence of
systems losing energy according to the prediction of gravitational-wave
emission from general relativity~\cite{Taylor:1982,Abadie:2010p10836}. As
energy is lost to gravitational waves, a binary's orbital separation decreases,
as does its orbital period.
As this occurs, the gravitational-wave amplitude increases, causing
a run-away inspiral that eventually leads to the merger of the compact objects.
One can estimate the expected coalescence rate by extrapolating the orbital
periods of systems observed electromagnetically. Although this procedure has
large uncertainty, tens of binary coalescence events are realistically expected
to be observed in the advanced detector era later this
decade~\cite{Abadie:2010p10836}.

The gravitational-wave inspiral signal starts at low frequencies, gradually
increasing in amplitude and frequency (i.e. it will chirp) until it reaches the
innermost stable circular orbit (\textsc{isco}) ($f_\mathrm{ISCO} \approx
4.4\times 10^3\,$Hz$\,\times(M/\Msun)^{-1}$) \editorial{Is $f_\mathrm{ISCO}$
the orbital frequency or GW frequency?} for a test particle orbiting a
Schwarzschild black hole with the total mass of the system $M$.  The frequency
response of Advanced ground-based gravitational-wave detectors is such that the
gravitational-wave frequency must be above $\sim 10$Hz before any appreciable
signal-to-noise ratio (\textsc{snr}) can be accumulated\footnote{Which implies
an orbital frequency of 5 Hz assuming that the dominant gravitational-wave
frequency is twice the orbital frequency}.  Most of the \textsc{snr} is
accumulated near the minimum of the detector noise spectral density at $\sim
100$Hz. The inspiral waveform of a binary neutron star system will spend $\sim
10^3$s in band.  Conveniently, the gravitational waveforms of compact binary
coalescence are understood to the level required by Advanced \textsc{ligo} to
detect~\cite{BuonannoIyerOchsnerYiSathya2009} via matched filtering techniques.
Matched filtering detection pipelines maximize signal-to-noise ratio over a
bank of filters, which are template waveforms in the binary parameter space.
The templates are spaced to bound the loss in signal-to-noise ratio for any
binary signal within the parameter space
boundaries~\cite{Owen:1995tm,Owen:1998dk}.

As a binary containing a neutron star with a low-mass companion (a neutron star
or black hole with mass less than $10\Msun$) approaches the merging point, the
neutron star or neutron stars are tidally disrupted. This can provide
the matter required to produce bright electromagnetic counterparts in the
moments following the merger~\cite{shibata:2007}.
\editorial{nvf: Chad thinks we need more mechanisms. I think this stands on
its own without the acceleration mechanism.}
In order to observe the prompt and potentially brightest
electromagnetic emission, telescopes will have to be pointing in the direction
of the event during the seconds surrounding the merger. In order to assure
that this occurs more often than by random chance from various astronomical
surveys operating independently, the gravitational-wave community has initiated
a project to send alerts when potential gravitational-wave transients are
observed. Unfortunately, at best, gravitational-wave experiments can only hope
to have accumulated sufficient \textsc{snr} within seconds (or less) prior to
the merger.  This makes it challenging to alert telescopes promptly to catch as
much of the electromagnetic signal as possible.

In 2010, \textsc{ligo} completed its sixth science run (S6) and Virgo
completed its third science run (VSR3). When both \textsc{ligo} and Virgo were
operating, several all-sky detection pipelines operated in a low-latency
configuration~\cite{HugheyGWPAW2011, S6lowlatency}. Several candidates were sent
for electromagnetic followup to several telescopes including Swift,
\textsc{rotse}, \textsc{tarot}, and Zadko~\cite{kanner2008, HugheyGWPAW2011}.
The S6 analyses achieved latencies of $~\sim30$--$60$ minutes, which were
dominated by
a human vetting process. The intrinsic gravitational-wave trigger generator
latency was $\sim$2--5 minutes. In this work we will assume that the human
vetting process can be automated in the advanced detector era and that the
dominating latency will be trigger generation. From this perspective we will
try to address the realistic expectations for reducing trigger latency from
$\sim$minutes to $\lesssim$seconds.

The reduction in latency is a challenging task.  As one pushes typical
finite-impulse-response (\textsc{fir}) filtering techniques from high latency
configurations with a computational complexity that scales as
$\mathcal{O}(N\log{N})$, where $N$ is the number of sample points in the
calculation, to low-latency configuration, the complexity goes to
$\mathcal{O}(N^2)$. Na\"ively this would require orders of magnitude more
computational resources. However, there are aspects of searches for binary
coalescence that can be exploited to offset some of the difference in
computational complexity. 1) The time-frequency morphology of the binary
waveforms is such that multi-rate filtering may be employed. 2) The redundancy
in the filter banks used to cover a particular parameter space may be
reduced via rank-reduction schemes such as the singular value
decomposition~\cite{Cannon:2010p10398}. We will use both aspects to
demonstrate that a very low latency analysis of gravitational-wave data for
compact binary sources is possible with the current computing resources.
Assuming other aspects of gravitational-wave observation latency
can be reduced significantly, this should allow the possibility for alerts to
be sent within seconds of the binary merger.

The paper is organized as follows. First we provide an overview of the
standard method for detecting compact binary coalescence signals and
describe how it can be modified to accomodate low-latency analysis. We
then describe the pipeline we have constructed that implements these changes.
To validate the approach we present results of simulations and conclude
with some remarks on what remains to prepare for the Advanced detector era.

