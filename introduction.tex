\section{Introduction}
\label{sec:introduction}

% Outline:
%  - Low-mass inspirals are the most promising source of gravitational radiation for advanced \textsc{ligo}
% - Motivation for high throughput
%   - Point out challenges of advanced detectors: more templates, longer templates
% - Motivation for low latency
%   - Observational applications of near-realtime detection
%   - Mention S6 and VSR2, online running, EM followup
% - Layout of paper: description of method, procedure followed 

\editorial{Would the introduction be more effective without this paragraph?} The coalescence of compact binary systems consisting of neutron stars (NS)
and/or black holes (BH) is the most promising source of gravitational radiation
for Advanced \LIGO~\cite{ALIGOWeb}, Virgo~\cite{AVirgoWeb}, \GEO~\cite{GEOWeb}, and
\LCGT~\cite{LCGTWeb}.  Tens of
binary coalescence events are expected to be observed in the
advanced detector era later this decade~\cite{Abadie:2010p10836}.

As a compact binary system loses energy to gravitational waves, its orbital separation
decreases. This causes a run-away inspiral with the gravitational-wave
amplitude and frequency increasing until the system eventually merges near the
innermost stable circular orbit (\ISCO).  If a neutron star is involved it may become tidally disrupted near the merger.
This disrupted matter can fuel a bright electromagnetic
counterpart in the system's final moments as a binary~\cite{shibata:2007}.
In order to observe the prompt and
potentially most intense electromagnetic emission, telescopes must point during the seconds surrounding the
merger.

\editorial{Citation needed for LOOC-UP} To make this possible, the gravitational-wave community initiated a project to send alerts when
potential gravitational-wave transients are observed.  In October 2010, \textsc{ligo} completed its sixth science run (S6) and Virgo
completed its third science run (VSR3).  While both \textsc{ligo} detectors and Virgo were
operating, several all-sky detection pipelines operated in a low-latency
configuration, namely \textsc{mbta}, ihope, Coherent WaveBurst, and Omega~\cite{HugheyGWPAW2011, S6lowlatency}. \editorial{Get references for these low-latency pipelines.} The S6 analysis achieved latencies of 30--60 minutes, which were
dominated by a human vetting process. Candidates were sent for
electromagnetic followup to several telescopes; Swift,
\textsc{rotse}, \textsc{tarot}, and Zadko~\cite{kanner2008, HugheyGWPAW2011} took images of likely sky locations.  \textsc{mbta} achieved the best gravitational-wave trigger
generation latencies of 2--5 minutes.  We assume that
in the advanced detector era the vetting process will be automated, so current
trigger generation and telescope actuation would then dominate latency.

To this end, we have the ambition of reporting candidates not minutes \emph{after} the merger, but seconds \emph{before}.  By looking for threshold crossings before the signal leaves the detection band, it is possible to trade some signal to noise ratio (\textsc{snr}) for latency.  Figure \ref{fig:earlywarning} shows projected early trigger rates for NS--NS binaries in Advanced \LIGO\
assuming the event rate predictions in \cite{Abadie:2010p10836}.
%
\begin{figure}
\includegraphics{N_before_Tc.pdf}
\caption{\label{fig:earlywarning} Expected number of NS--NS sources that will
be detectable $T_{bc}$ seconds before coalescence.  The solid line is the most
likely yearly rate estimate $N_{\mathrm{re}}$ for Advanced \LIGO\ and the
shaded region is the interval $N_{\mathrm{low}}$ to $N_{\mathrm{high}}$ from
\cite{Abadie:2010p10836}.  Note that assuming \SNR\ 8 is sufficient for
detection and that we observe $N_{\mathrm{re}} = 40$ events per year with a
detector having the ZERO\_DET\_high\_P noise model described
in~\cite{ALIGONoiseZERO_DET_high_P}, $\sim10$ sources may be detected within 10
seconds of merger and $\sim1$ sources maybe detected within 100 seconds of
merger.  Other noise models, ZERO\_DET\_low\_P~\cite{ALIGONoiseZERO_DET_low_P}
and BHBH\_20deg~\cite{ALIGONoiseBHBH_20deg} provide better results for early
detection but at the cost of fewer total events observed above SNR 8.}
\end{figure}

Although it is technically possible to detect NS--NS signals with the
\earlywarning\ predicted in figure \ref{fig:earlywarning}, the reduction in
latency is a challenging task.  As finite-impulse-response (\fir) filtering
techniques are pushed from high latency configurations to low-latency
configurations more computational resources are required.  A brute-force
implementation of low lateny filtering may be too computationally costly to
implement.  However, there are aspects of searches for binary coalescence that
can be exploited to offset some of the intrinsic computational complexity: 1)
The time-frequency evolution of the binary waveforms is such that multi-rate
filtering may be employed. 2) The redundancy in the filter banks used to cover
a particular parameter space may be reduced via rank-reduction schemes such as
the singular value decomposition~\cite{Cannon:2010p10398}. We will use both
aspects to demonstrate that a very low latency analysis with predictive
detection of compact binary sources is possible with current computing
resources.  Assuming other aspects of gravitational-wave observation latency
can be reduced significantly, this should allow the possibility for prompt
alerts to be sent to the Astronomical community.

The paper is organized as follows. First we provide an overview of our method
for detecting compact binary coalescence signals in an \earlywarning\
analysis. We then describe the pipeline we have constructed that implements
our method.  To validate the approach we present results of simulations and
conclude with some remarks on what remains to prepare for the Advanced
detector era.

