\section{Introduction}
\label{sec:introduction}
Modern gravitational-wave observatories are likely to deliver on the promise 
that their name implies by observing multiple gravitational-wave signals from
compact binary coalescence within the coming decade ~\cite{kopparapu2008}. 
There is now a world-wide network of gravitational-wave observatories 
including the laser interferometers
LIGO, VIRGO, GEO, and TAMA, which should assure that coincident detections 
will be possible among at least three separate sites lending confidence
to any detections.
~\cite{Abbott:2005qm, beauville2008, beauville2006, blackburn2005, LIGOS1instpaper}.  
However, having an electromagnetic or neutrino counterpart to a 
gravitational-wave detection would not only increase the confidence in 
the detection but would also greatly improve the astrophysical information
available from the event.

Work has already commenced to trigger gravitational-wave searches from 
electromagnetic observations ~\cite{triggeredsearches2008}. LIGO has recently
ruled out the possibility of GRB070201 originating from the merger of a
neutron star / neutron star or neutron star / black hole system in the
Andromeda galaxy ~\cite{GRB070201}.  The still open question is, will 
gravitational-wave observations be able to trigger electromagnetic followup?


This question has been investigated and it seems  at least
possible that gravitational-wave detections could provide a region of the
sky to prompt electromagnetic observation followups~\cite{sylvestre2003}.
Much work is currently underway in providing the best possible source 
localization from gravitational-wave detector networks
~\cite{markowitz2008, raymond2008, cavalier2006}.  Colloborations are
currently forming to provide infrastructure for the gravitational-wave
Astronomers to provide targets of opportunity for electromagnetic 
Astronomers ~\cite{kanner2008}.  Most models that predict simultaneous
gravitational-wave and electromagnetic observations also predict that the
peak amplitude of electromagetic radiation will occur soon after 
gravitational-wave emmission ~\cite{sylvestre2003}.  Thus in order to maximize
the chance of a successful electromagnetic followup the latencey of 
gravitational-wave signal analysis must be made minimal.

The parameter space of compact binary coalescence signals is large
~\cite{Owen:1995tm, Owen:1998dk} .  It is a
computationally burdensome task to analyze these signals with even moderate
latency ~\cite{Abbott:2007xi}.  
This work will describe how to exploit degeneracy in the signal
parameter space to answer more quickly whether or not a gravitational-wave 
is present, with what confidence, and where it is coming from. Specifically
we will explore using the singular value decomposition (SVD) to reduce the
effective number of filters required to search the data.  We note that others
have applied the use of SVD to gravitational wave data analysis to analyze
optimal gravitational-wave burst detection 
~\cite{bradyraymajumder2004, heng2008} and coherent networks 
of detectors ~\cite{wen2008} 

The paper is organized as follows.  First we provide an overview of the 
standard method for detecting compact binary coalescence signals and
describe how it can be modified to accomodate low latency analysis.  We
then describe the pipeline we have constructed to implement these changes.
To validate the approach we present results of simulations and finish
with some concluding remarks. 
