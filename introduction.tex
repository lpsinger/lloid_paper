\section{Introduction}
\label{sec:introduction}

% Outline:
%  - Low-mass inspirals are the most promising source of gravitational radiation for advanced \textsc{ligo}
% - Motivation for high throughput
%   - Point out challenges of advanced detectors: more templates, longer templates
% - Motivation for low latency
%   - Observational applications of near-realtime detection
%   - Mention S6 and VSR2, online running, EM followup
% - Layout of paper: description of method, procedure followed 


The coalescence of compact binary systems consisting of neutron stars and/or
black holes is the most promising source of gravitational radiation for
ground-based gravitational wave detectors such as Advanced
\textsc{ligo}~\cite{ALIGOWeb}, Virgo~\cite{AVirgoWeb}, GEO~\cite{GEOWeb}, and
LCGT~\cite{LCGTWeb}. Radio observation of binary pulsar systems in the Galaxy
show that such systems are losing energy according to the prediction of
gravitational wave emission from general
relativity~\cite{Taylor:1982}\citeneeded.  The binary periods are decreasing as
they lose energy to gravitational waves and subsequently the separation of the
component neutron stars is also decreasing. As this occurs, the gravitational
wave amplitude increases causing a run-away reaction that will eventually lead
to the merger of the pulsars within the age of our Universe.  One can
extrapolate the expected coalescence rate from the electromagnetically observed
systems.  Although this procedure has large uncertainty, tens of binary
coalescence events are realistically expected to be observed in the advanced
detector era later this decade~\cite{Abadie:2010p10836}.

Only the final tens of minutes of the merger will produce a signal with
sufficient amplitude that it will be observable with Advanced LIGO\citeneeded.
Furthermore, due to the frequency response of Advanced ground based
gravitational wave detectors the last tens of seconds produce the majority of
the signal-to-noise ratio (SNR).  The gravitational-wave frequency must be
above $\sim$10Hz  before any appreciable signal-to-noise ratio can be
accumulated\footnote{Which implies an orbital frequency of 5 Hz assuming that
the dominant gravitational wave frequency is twice the orbital frequency} and
most of the SNR is accumulated near the minimum of the detector noise spectral
density $\sim100$Hz.  The gravitational wave signal will increase in amplitude
and frequency (i.e. it will chirp) until it reaches the innermost stable
circular orbit (ISCO), which is $\approx 4.4\times
10^3\,$Hz$\,\times(M/\Msun)^{-1}$ for a test particle orbiting a Schwarzschild
black hole with mass $M$. After it reaches the ISCO the binary will merge, and
if it forms a black hole will emit gravitational radiation according to the
fundamental modes of the resulting Kerr black hole (which are exponentially
decreasing in time)\citeneeded.  This entire process is sufficiently well
modeled that matched filtering can be used to detect these systems by
establishing a grid of filters in the binary parameter space in a way that
assures minimal loss in signal-to-noise ratio for any binary signal within the
parameter space boundaries\citeneeded.

As the binary approaches the merging point, the neutron stars are tidally
disrupted.  This may provide the matter required to produce bright
electromagnetic counterparts in the moments before and during the
merger~\cite{shibata:2007}.   \hl{List more mechanisms!! such as plasma and
magnetic fields...}.  In order to observe the prompt and potentially brightest
electromagnetic emission, telescopes will have to be pointing in the direction
of the event during the seconds surrounding the merger.  In order to assure
that this occurs more often than by random chance from various astronomical
surveys operating independently, the gravitational-wave community has initiated
a project to send alerts when potential gravitational wave transients are
observed.  Unfortunately, at best, gravitational wave experiments can only hope
to have accumulated sufficient SNR within seconds (or less) prior to the merger.
This makes it challenging to alert telescopes promptly to catch as much of
the electromagnetic signal as possible.  

%
\editorial{If we publish before the LUMIN papers, we have to redo this section.
We can't make claims about those private communications with the telescopes.
Else, cite the pre-S6 LUMIN paper \cite{kanner2008}. CH, the info is public,
Brennan presented it at GWPAW, fair game then.  I have cited the presentation
for now and spoke with Brennan about the upcoming paper (also cited as in
preparation)} 
%
In \oldstylenums{2010}, LIGO completed its sixth science run (S6) and Virgo
completed its third science run (VSR3).   When both LIGO and Virgo were
operating, several all-sky detection pipelines operated in a low-latency
configuration~\cite{HugheyGWPAW2011, S6lowlatency}. A few candidates were sent
for electromagnetic followup to several telescopes including Swift,
\textsc{rotse}, \textsc{tarot}, and Zadko~\cite{kanner2008, HugheyGWPAW2011}.
The S6 analyses achieved latencies of $~\sim$30 minutes, which were dominated by
a human vetting process.  The intrinsic gravitational-wave trigger generator
latency was $\sim$2--5 minutes.  In this work we will assume that the human
vetting process can be automated in the advanced detector era, and that the
dominating latency will be trigger generation.  From this perspective we will
try to address the realistic expectations for reducing trigger latency from
$\sim$minutes to $\lesssim$seconds.

The reduction in latency is a challenging task.  As one pushes typical
finite-impulse-response (FIR) filtering techniques from high latency
configurations with a computational complexity that scales as
$\mathcal{O}(N\log{N})$, where $N$ is the number of sample points in the
calculation, to low-latency configuration, the complexity goes to
$\mathcal{O}(N^2)$.  Naively this would require orders of magnitude more
computational resources.  However, there are aspects of searches for binary
coalescence that can be exploited to offset some of the difference in
computational complexity.  1) The time frequency properties of the binary
waveforms are such that multirate filtering may be employed. 2) The redundancy
in the filter banks used to cover a particular mass parameter space may be
reduced via rank reduction schemes such as singular value
decomposition~\cite{Cannon:2010p10398}.  We will use both aspects to
demonstrate that a very low latency analysis of gravitational-wave data for
compact binary sources is possible with the current computing resources
available.  Assuming other aspects of gravitational-wave observation latency
can be reduced significantly, this should allow the possibility for alerts to
be sent within seconds of the binary merger.

The paper is organized as follows.  First we provide an overview of the 
standard method for detecting compact binary coalescence signals and
describe how it can be modified to accomodate low latency analysis.  We
then describe the pipeline we have constructed to implement these changes.
To validate the approach we present results of simulations and finish
with some concluding remarks.

