\section{Introduction}
\label{sec:introduction}

% Outline:
%  - Low-mass inspirals are the most promising source of gravitational radiation for advanced \textsc{ligo}
% - Motivation for high throughput
%   - Point out challenges of advanced detectors: more templates, longer templates
% - Motivation for low latency
%   - Observational applications of near-real-time detection
%   - Mention S6 and VSR2, online running, EM followup
% - Layout of paper: description of method, procedure followed 

As a compact binary system loses energy to gravitational waves (\GW{}s), its
orbital separation decays, leading to a run-away inspiral with the \GW{}
amplitude and frequency increasing until the system eventually merges soon
after the innermost stable circular orbit (\ISCO). If a neutron star is
involved it may become tidally disrupted near the merger.  This disrupted
matter can fuel a bright electromagnetic counterpart in the merger's
aftermath~\cite{shibata:2007}.

\GW{} detectors are sensitive the to these \GW{}
signals.  Indeed, the coalescence of compact binary systems consisting of
neutron stars (\textsc{ns}) and/or black holes (\textsc{bh}) is the most
promising source of gravitational radiation for Advanced \LIGO, Virgo, \GEO\,
and \LCGT~\cite{ALIGOWeb, AVirgoWeb, GEOWeb, LCGTWeb}.  Tens of binary
coalescence events per year are expected to be observed in the advanced
detector era later this decade~\cite{Abadie:2010p10836}. In the era of
first-generation detectors the \GW{} community initiated a project
to send alerts when potential \GW{} transients were observed in
order to trigger followup observations by optical telescopes.  The best
demonstrated latencies were 30--60 minutes~\cite{HugheyGWPAW2011}, which was an
important achievement, but far too late to catch prompt electromagnetic
emission.

Prompt electromagnetic emission can arise as shells of relativistically
out-flowing matter collide in the inner shock. Such an inner shock from a
compact binary coalescence is believed to be a mechanism for short gamma-ray
bursts (short \GRB{}s)~\cite{Lee:2005, nakar07}. The same inner shocks, or
potentially reverse shocks, can produce a bright accompanying optical
flash~\cite{Sari99}. Prompt emission is a probe into the extreme initial
conditions of the outflow, in contrast with afterglows, which arise in the
external shock with the local medium and are relatively
insensitive to initial conditions. Optical flashes have only been observed to
peak within tens of seconds for a handful of long
\GRB{}s~\cite{2011CRPhy..12..255A} by telescopes with extremely rapid response
or, in the case of \textsc{grb 080319b}, by pure serendipity, where several
telescopes were observing a previous \GRB{} in the same
field~\cite{2008Natur.455..183R}. Short \GRB{}s, on the other hand, typically
fade too quickly to even catch the tails of the afterglows. Rapid
\GW{} transient alerts could enable the first observation of
prompt optical flashes and the rise of afterglows from short \GRB{}s, and, for
an exceptional event, perhaps even a glimpse of a
precursor~\cite{0004-637X-723-2-1711}. An electromagnetic counterpart would
vastly boost confidence in the \GW{} detection and provide the
tight sky localization necessary to allow determination of the source's host
galaxy, which leads to a redshift measurement. A coincident \GW{}
observation will provide a luminosity distance measurement. With these
together, we can produce precision measurements of the Hubble
constant~\cite{2010ApJ...725..496N}.

To this end, we have the ambition of reporting \GW{} candidates not minutes
\emph{after} the merger, as has already been accomplished, but seconds
\emph{before}.  By examining the signal to noise ratio (\SNR{}) stream for
threshold crossings before the \GW{} signal leaves the detection band, it is
possible to trade some \SNR{} and sky localization
accuracy~\cite{Fairhurst2009} for latency.  Figure \ref{fig:earlywarning} shows
projected early trigger rates for \textsc{ns}--\textsc{ns} binaries in Advanced
\LIGO\ assuming the event rate predictions in~\cite{Abadie:2010p10836}.  The
most likely estimates indicate that 10 sources could be audible (meaning
producing an \SNR\ above 8) 10 seconds or more prior to the merger during one
year of observation.  This number is highly uncertain.  In fact as many as 100
events detected 10 seconds early is plausible. It is also plausible that no
sources will be detected at all.  The gray bands give the pessimistic to
optimistic interval defined in ~\cite{Abadie:2010p10836}. 
%
\begin{figure}
\begin{center}
\includegraphics{figures/snr_in_time.pdf}
\caption{\label{fig:earlywarning} Expected number of \textsc{ns}--\textsc{ns}
sources that will be detectable $t$ seconds before coalescence.  The heavy
solid line is the most likely yearly rate estimate $\dot N_{\mathrm{re}}$
during Advanced \LIGO.  The shaded region represents the confidence interval
$[\dot N_{\mathrm{low}}, \dot N_{\mathrm{high}}]$ described in
\cite{Abadie:2010p10836}.  Advanced \LIGO\ noise projections are described in
\cite{ALIGONoise}.  Assuming that an \SNR\ of 8 is sufficient for detection and
that we observe $\dot N_{\mathrm{re}} = 40$ events~yr$^{-1}$ with a detector in
the `zero detuning, high power' configuration, about 10~yr$^{-1}$ will be
detectable within 10 seconds before merger and $\sim1$~yr$^{-1}$ will be
detectable within 100 seconds of merger.} 
\end{center}
\end{figure}

In October 2010, \LIGO{} completed its sixth science run
(S6) and Virgo completed its third science run (VSR3).  While both
\LIGO{} detectors and Virgo were operating, several all-sky detection
pipelines operated in a low-latency configuration to send astronomical alerts,
namely \textsc{mbta}, Coherent WaveBurst, and
Omega~\cite{HugheyGWPAW2011, S6lowlatency}.
\editorial{Get references for these low-latency pipelines.} The S6 analyses
achieved latencies of 30--60 minutes, which were dominated by a human vetting
process. Candidates were sent for electromagnetic followup to several
telescopes; Swift, \textsc{rotse}, \textsc{tarot}, and Zadko~\cite{kanner2008,
HugheyGWPAW2011} took images of likely sky locations.  \textsc{mbta} achieved
the best \GW{} trigger-generation latencies of 2--5 minutes.  We
assume that in the advanced detector era the vetting process will be automated,
so current \GW{} search methodology and telescope actuation would
dominate latency.

Advance detection of compact binary coalescences (\CBC{}s) will require striking a balance between latency
and throughput.  \CBC{} searches consist of banks of matched filters, or
cross-correlations between the data stream and a bank of nominal ``template''
signals.  There are many different implementations of matched filters, but most
have high throughput at the cost of high latency, or low latency at the cost of
low throughput.  The former are epitomized by the overlap-save algorithm
\cite{numerical-recipes-chapter-13} for frequency domain (\FD) convolution,
currently the preferred method in \GW{}
searches.  The most obvious example of the latter is the time domain
(\TD) convolution, which has no algorithmic latency.  However, its
computational complexity is quadratic in the length of the templates, so it is
prohibitively expensive for long templates.

Fortunately, the morphology of inspiral signals can be exploited to offset some
of the computational complexity of low-latency algorithms.  First, the signals
evolve slowly in frequency, so that they can be broken into contiguous
band-limited time intervals and processed at possibly lower sample rates.
Second, inspiral filter banks consist of highly similar templates, admitting
methods such as singular value decomposition (\SVD{}) to reduce the number of
templates \cite{Cannon:2010p10398}. We will use both aspects to demonstrate that a very
low latency analysis with advance detection of compact binary sources is
possible with current computing resources.  Assuming other technical sources of
latency can be reduced significantly, this should allow the possibility for
prompt alerts to be sent to the astronomical community.

The paper is organized as follows. First we provide an overview of our method
for detecting compact binary coalescence signals in an \earlywarning\ analysis.
We then describe the pipeline we have constructed that implements our method.
To validate the approach we present results of simulations and conclude with
some remarks on what remains to prepare for the advanced detector era.

