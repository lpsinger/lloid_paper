\section{Introduction}
\label{sec:introduction}

% Outline:
%  - Low-mass inspirals are the most promising source of gravitational radiation for advanced \textsc{ligo}
% - Motivation for high throughput
%   - Point out challenges of advanced detectors: more templates, longer templates
% - Motivation for low latency
%   - Observational applications of near-realtime detection
%   - Mention S6 and VSR2, online running, EM followup
% - Layout of paper: description of method, procedure followed 

%
The coalescence of compact binary systems consisting of neutron stars (NS)
and/or black holes (BH) is the most promising source of gravitational radiation
for ground-based gravitational-wave detectors such as Advanced
\LIGO~\cite{ALIGOWeb}, Virgo~\cite{AVirgoWeb}, \GEO~\cite{GEOWeb}, and
\LCGT~\cite{LCGTWeb}.  Detections will constrain the mass and spin of compact
objects~\cite{finn1993,Poisson:1995ef}, the neutron star equation of
state~\cite{flanagan:021502,Read:2009}, general relativity in the strong-field
regime~\cite{Will:2005va}, the graviton
mass~\cite{PhysRevD.80.044002,Keppel:2010qu}, and the Lorentz Invariance
principle~\cite{Ellis2006402}.  

Radio observation of binary pulsar systems in the Galaxy provide indirect
evidence that such systems are losing energy according to the prediction of
gravitational-wave emission from general
relativity~\cite{Taylor:1982,Abadie:2010p10836}.  One can estimate the expected
coalescence rate by extrapolating the orbital periods of systems observed
electromagnetically.  Although this procedure has large uncertainty, tens of
binary coalescence events are realistically expected to be observed in the
advanced detector era later this decade~\cite{Abadie:2010p10836}.

As a binary system loses energy to gravitational waves, its orbital separation
decreases. This causes a run-away inspiral with the gravitational-wave
amplitude and frequency increasing until the system eventually merges near the
innermost stable circular orbit (\ISCO).
%
\footnote{$f_\mathrm{ISCO} \approx 4.4\times 10^3\,$Hz$\,\times(M/\Msun)^{-1}$. 
for a test particle orbiting a Schwarzschild black hole with the total mass of
the system $M$.}
%
If a neutron star is involved it may become tidally disrupted near the merger.
This can provide the matter required to produce bright electromagnetic
counterparts in the moments following the merger~\cite{shibata:2007}.
\editorial{nvf: Chad thinks we need more mechanisms. I think this stands on its
own without the acceleration mechanism.} In order to observe the prompt and
potentially brightest electromagnetic emission, telescopes will have to be
pointing in the direction of the event during the seconds surrounding the
merger. In order to assure that this occurs more often than by random chance
from various astronomical surveys operating independently, the
gravitational-wave community has initiated a project to send alerts when
potential gravitational-wave transients are observed. 

In 2010, \textsc{ligo} completed its sixth science run (S6) and Virgo
completed its third science run (VSR3). When both \textsc{ligo} and Virgo were
operating, several all-sky detection pipelines operated in a low-latency
configuration~\cite{HugheyGWPAW2011, S6lowlatency}. Candidates were sent for
electromagnetic followup to several telescopes including Swift,
\textsc{rotse}, \textsc{tarot}, and Zadko~\cite{kanner2008, HugheyGWPAW2011}.
The S6 analyses achieved latencies of $~\sim30$--$60$ minutes, which were
dominated by a human vetting process. The intrinsic gravitational-wave trigger
generator latency was $\sim$2--5 minutes.  In this work we will assume that
the human vetting process can be automated in the advanced detector era and
that the dominating latency will be trigger generation. From this perspective
we will try to address the realistic expectations for reducing trigger latency
from $\sim$ minutes \emph{after} the merger to minutes \emph{before} the
merger for sufficiently loud events.  Figure \ref{fig:earlywarning} presents
the possibility for early detection of NS--NS binaries with Advanced \LIGO\
assuming the rate predictions in \cite{Abadie:2010p10836}.
%
\begin{figure}
\includegraphics{N_before_Tc.pdf}
\caption{\label{fig:earlywarning}\FIXME{Drew has independently reproduced
this calculation} Expected number of NS--NS sources that will be detectable
$T_{bc}$ seconds before coalescence.  The solid line is the most likely yearly
rate estimate $N_{\mathrm{re}}$ for Advanced \LIGO\ and the shaded region is
the interval $N_{\mathrm{low}}$ to $N_{\mathrm{high}}$ from
\cite{Abadie:2010p10836}.  Note that assuming \SNR\ 8 is sufficient for
detection and that we observe $N_{\mathrm{re}} = 40$ events per year with a
detector having the ZERO\_DET\_high\_P noise model described
in~\cite{ALIGONoiseZERO_DET_high_P}, $\sim10$ sources may be detected within
10 seconds of merger and $\sim1$ sources maybe detected within 100 seconds of
merger.  Other noise models, ZERO\_DET\_low\_P~\cite{ALIGONoiseZERO_DET_low_P}
and BHBH\_20deg~\cite{ALIGONoiseBHBH_20deg} provide better results for early detection
but at the cost of fewer total events observed above SNR 8.}
\end{figure}

Although it is technically possible to detect NS--NS signals with the
\earlywarning\ predicted in figure \ref{fig:earlywarning}, the reduction in
latency is a challenging task.  As finite-impulse-response (\fir) filtering
techniques are pushed from high latency configurations to low-latency
configurations more computational resources are required.  A brute-force
implementation of low lateny filtering may be too computationally costly to
implement.  However, there are aspects of searches for binary coalescence that
can be exploited to offset some of the intrinsic computational complexity: 1)
The time-frequency evolution of the binary waveforms is such that multi-rate
filtering may be employed. 2) The redundancy in the filter banks used to cover
a particular parameter space may be reduced via rank-reduction schemes such as
the singular value decomposition~\cite{Cannon:2010p10398}. We will use both
aspects to demonstrate that a very low latency analysis with predictive
detection of compact binary sources is possible with current computing
resources.  Assuming other aspects of gravitational-wave observation latency
can be reduced significantly, this should allow the possibility for prompt
alerts to be sent to the Astronomical community.

The paper is organized as follows. First we provide an overview of our method
for detecting compact binary coalescence signals in an \earlywarning\
analysis. We then describe the pipeline we have constructed that implements
our method.  To validate the approach we present results of simulations and
conclude with some remarks on what remains to prepare for the Advanced
detector era.

