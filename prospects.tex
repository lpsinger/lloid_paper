\section{Prospects for \earlywarning\ detection and \EM\ followup}

\begin{figure}[h]
\plotone{f1}
\caption{\label{fig:earlywarning}Expected number of \NS--\NS{}
sources that could be detectable by Advanced \LIGO\ a given number of seconds
before coalescence.  The heavy solid line is the most realistic yearly rate
estimate.  The shaded region represents the 5 to 95\% confidence interval
arising from uncertainty in predicted event rates \citep{Abadie:2010p10836}.}
\end{figure}
%
Before the \GW{} signal leaves the detection band, we can imagine examining the
signal to noise ratio (\SNR{}) accumulated up to that point and if it is
already significant, release an alert immediately, trading \SNR{} and sky
localization accuracy for pre-merger detection.
Figure~\ref{fig:earlywarning} shows projected early detectability rates for
\NS--\NS{} binaries in Advanced \LIGO\ assuming event rates
predicted in \citet{Abadie:2010p10836} and anticipated detector sensitivity for
the `zero detuning, high power' configuration described in \citet{ALIGONoise}.
The most realistic estimates indicate that at an \SNR\ threshold of 8, we will
observe a total of 40~events~yr$^{-1}$. $\sim$10~yr$^{-1}$ will be detectable
within 10~s of merger and $\sim$5~yr$^{-1}$ will be detectable within 25~s of
merger if analysis can proceed with near zero latency. These rates are shown
with their substantial uncertainties in Figure~\ref{fig:earlywarning}.

We emphasize that any practical \GW\ search will include technical delays due
to light travel time between the detectors, detector infrastructure, and the
selected data analysis strategy.  Figure~\ref{fig:earlywarning} must be understood
in the context of all of the potential sources of latency, some of which are avoidable
and some of which are not. 


\begin{table}[h]
\caption{\label{table:sky-localization-accuracy}Horizon distance, \SNR\ at
merger, and area of 90\% confidence at selected times before merger for sources
with expected detectability rates of 40, 10, 1, and 0.1~yr$^{-1}$.}
\begin{center}
\begin{tabular}{rrrrrrr}
\tableline\tableline
rate & horiz. & final & \multicolumn{4}{c}{$A$(90\%) (deg$^2$)} \\
\cline{4-7}
yr$^{-1}$ & (Mpc) & \SNR\ & 25 s & 10 s & 1 s & 0 s \\
\tableline
40\phd\phn & 445 & 8.0 & 15500 & 3100 & 200 & 9.6 \\
10\phd\phn & 280 & 12.7 & 6100 & 1200 & 78 & 3.8 \\
1\phd\phn & 130 & 27.4 & 1300 & 260 & 17 & 0.8 \\
0.1 & 60 & 58.9 & 280 & 56 & 3.6 & 0.2 \\
\tableline
\end{tabular}
\end{center}
\end{table}
%
\begin{figure}[h]
\plotone{f2}
\caption{\label{fig:sky-localization-accuracy}Area of the 90\% confidence
region as a function of time before coalescence for sources with anticipated
detectability rates of 40, 10, 1, and 0.1~yr$^{-1}$. The heavy dot indicates
the time at which the accumulated \SNR\ exceeds a threshold of~8.}
\end{figure}

\EM\ followup requires estimating the location of the \GW{} source. The localization
uncertainty can be estimated from the uncertainty in the time of arrival of the \GW{}s,
which is determined by the signal's effective bandwidth and \SNR{}
\citep{Fairhurst2009}.  Table~\ref{fig:earlywarning} and
Figure~\ref{fig:sky-localization-accuracy} show the estimated minimum 90\%
confidence area versus time of the loudest coalescence events detectable by
Advanced \LIGO{} and Advanced Virgo.  Once per year, we expect to observe an
event with a final \SNR\ of $\approx$27 whose location can be constrained to about
1300~deg$^2$ (3.1\% of the sky) within 25~s of merger,
260~deg$^2$ (0.63\% of the sky) within 10~s of merger, and
0.82~deg$^2$ (0.0020\% of the sky) at merger.

It is unfeasible to search hundreds of square degrees for a prompt counterpart.
For reference, LSST should see first light around the same time as
Advanced \LIGO{} and will have an unparalleled field of view of 9.6~deg$^2$
\citep{2008arXiv0805.2366I}.  However, 
it is possible to reduce the localization uncertainty by only looking at
galaxies from a catalog that lie near the sky location and luminosity distance
estimate from the \GW{} signal~\citep{galaxy-catalog} as was done in S6/VSR3.
Within the expected Advanced \LIGO{} \NS--\NS{} horizon distance,
the number of galaxies that can produce a given signal amplitude is much larger
than in Initial \LIGO{} and thus the catalog will not be as useful
for downselecting pointings for most events. However, exceptional \GW{} sources will
necessarily be extremely nearby. Within this reduced volume there will be fewer
galaxies to consider for a given candidate and catalog completeness will be
less of a concern.  This should reduce the 90\% confidence area substantially.
Proposed third generation \GW{} detectors would improve the \SNR\ and likewise
sky localization.
