\documentclass[letterpaper,11pt]{article}
\usepackage{amssymb}
\usepackage{amsmath}
\usepackage{stmaryrd}  % gives \shortuparrow and \shortdownarrow

\newcommand{\downsample}[1]{\ensuremath{\, \downarrow^{s-1} \star \,\, #1}}
\newcommand{\upsample}[1]{\ensuremath{\, \uparrow^{s+1} \star \,\, #1}}

\title{Notation proposal}
\author{Leo and Nick}

\begin{document}
\maketitle

We define a general discrete time series $p[k]$ with k indicating the time
samples at the maximum sample rate of the analysis.  We define a time series
$p^{s}[k^{s}]$ as being the time series $p[k]$ at a different (and lower)
sample rate $s$.  It is also necessary to thus define the operation that
changes the sample rate.

Decimation:
\begin{equation}
p^{s-1}[k^{s-1}] = \downsample{p^{s}[k^{s}]}
\end{equation}

Interpolation:
\begin{equation}
p^{s+1}[k^{s+1}] = \upsample{p^{s}[k^{s}]}
\end{equation}

In general downsampling and upsampling are a convolution of a antialiasing
filter or an interpolation function, respectively.  In the special case of a
band limited function, decimation may simply reduce to taking every nth sample
point, depending on the relative sample rates.

It is also necessary to introduce notation representing the time slices
of a template $\tau$

Definition of time sliced templates:
\begin{align}
h_{i}[k] &= \sum_{\tau} h_{i}^{\tau}[k] \\
h_{i}^{\tau}[k] &= \left\{ \begin{aligned}
    h_{i}[k] & \; \text{if } t^{\tau} < \frac{k}{f} \leq t^{\tau+1} \\
    0 & \; \text{, otherwise }
    \end{aligned} \right.
\end{align}

Since the templates are band limited an antialiasing filter is not required to
downsample, we need only to remove certain sample points as long as the desired
sample rate is at least twice the frequency content of the template in a time

The index $\tau$ when present will be used to denote a quantity that pertains
to a given time slice.

Definition of matched filter at the full sample rate:
\begin{equation}
\rho_i [k] = \sum_{n=0}^{N-1} h_{i}[n] x [k-n]
\end{equation}
where $h_i[k]$ denotes the $k^{\mathrm{th}}$ sample of the $i^{\mathrm{th}}$
template waveform and $x[k-n]$ represents the $(k-n)^{\mathrm{th}}$ sample of
the data.

Quantities involving the SVD are at a given sample rate $s$ and in a time slice
$\tau$

\textsc{svd}:
\begin{equation}
h^{\tau s}_{i}\!\left[k^{s}\right] = \sum_{m=0}^{M-1} v^{\tau s}_{im} \sigma^{\tau s}_m u^{\tau s}_{m}[k^{s}] \approx \sum_{l=0}^{L^s-1} v^{\tau s}_{il} \sigma^{\tau s}_l u^{\tau s}_{l}[k^{s}]
\end{equation}

Early-warning \textsc{snr}:
\newcommand{\mystrut}{\rule[-2em]{0pt}{0pt}}  % line up the underbraces
\begin{equation}
    \rho^{\tau s}_i[k^{s}] = \underbrace{\mystrut\sum_{l=0}^{L^{s}-1} v^{\tau s}_{il} \sigma^{\tau s}_l}_{\substack{\text{Reconstruction}\\ \text{with {\sc SVD}}}}
    \underbrace{\mystrut\sum_{n=0}^{N^{s} - 1} u^{\tau s}_{l}[n] x[k^{s}-n]}_{\substack{\text{orthogonal}\\ \text{{\sc FIR} filter}}} + 
    % + 
    % \underbrace{\mystrut\sum_{n=-N^\shortuparrow/2}^{N^\shortuparrow/2-1} b^\shortuparrow[n] \sum_{q \in \mathbb{Z}} \delta\!\left[k-n, q \frac{f^{s}}{f^{s+1}}\right]}_\text{interpolation filter} 
    \underbrace{\mystrut\rho^{\tau - 1 s}_i[k^{s}]}_{\substack{\text{accum.}\\ \text{prior {\sc snr}}}}
\end{equation}

\begin{table}
\begin{tabular}{rl}
\hline
\textbf{Symbol} & \textbf{Definition} \\\hline
$f^{s}$ & The sample rate for time-slice $s$\\
$t^{s}$ & The time of ``late'' boundary for time-slice $s$\\
$N^{s}$ & The number of samples in time-slice $s$\\
$L^{s}$ & The number of principal components in time-slice $s$\\
$N^\shortuparrow$ & The number of sample points in the interpolation filter\\
$N^\shortdownarrow$ & The number of sample points in the decimation filter\\
$M$ & The number of templates\\
\hline
\end{tabular}
\caption{constants}
\end{table}

\end{document}
