\documentclass[letterpaper,11pt]{article}
\usepackage{geometry}                % See geometry.pdf to learn the layout options. There are lots.
\usepackage{graphicx}
\usepackage{amssymb}
\usepackage{amsmath}

\title{Notation proposal}
\author{Leo and Nick}

\begin{document}
\maketitle

Definition of matched filter:
\begin{equation}
\rho_i [k] = \sum_{n=0}^{N-1} h_{i n} x [k-n]
\end{equation}

Definition of time slices:
\begin{align}
h_{i k} &= \sum_s h_{i k}^{(s)} \\
h_{i k}^{(s)} &= \left\{ \begin{aligned}
    h_{ik} & \; \text{if } t^{(s+1)} < \frac{k}{f} \leq t^{(s)} \\
    0 & \; \text{, otherwise }
    \end{aligned} \right.
\end{align}

Definition of decimation:
\begin{equation}
x^{(s+1)}[k] = \sum_{n=-N^\downarrow/2}^{N^\downarrow/2-1} b_n^\downarrow x^{(s)}\!\left[k \frac{f^{(s)}}{f^{(s+1)}} - n\right]
\end{equation}

\textsc{svd}:
\begin{equation}
h^{(s)}_{i\left(\frac{f}{f^{(s)}} k\right)} = \sum_{m=0}^{M-1} v^{(s)}_{im} \sigma^{(s)}_m u^{(s)}_{mk}
\end{equation}

Early-warning \textsc{snr}:
\begin{equation}
    \rho^{(s)}_i[k] = \underbrace{\sum_{l=0}^{L^{(s)}-1} v^{(s)}_{il} \sigma^{(s)}_l}_{\substack{\text{Reconstruction}\\ \text{with {\sc SVD}}}}
    \underbrace{\sum_{n=0}^{N^{(s)} - 1} u^{(s)}_{ln} x^{(s)}[k-n]}_{\substack{\text{orthogonal}\\ \text{{\sc FIR} filter}}} + \underbrace{\sum_{n=-N^\uparrow/2}^{N^\uparrow/2-1} b^\uparrow_n \sum_{q \in \mathbb{Z}} \delta\!\left[k-n, q \frac{f^{(s)}}{f^{(s-1)}}\right]}_\text{interpolation filter} \underbrace{\rho^{(s+1)}_i[q]}_{\substack{\text{accumulated {\sc snr}}\\ \text{from prior time slices}}}
\end{equation}

nvf commentary: I think it would be more natural to use $h_i[k]$ everywhere rather than $h_{ik}$. Accordingly, I'd notate $u^{(s)}_m[k]$ as the basis templates. Leo disagreed.

\begin{table}
\begin{tabular}{rl}
\hline
\textbf{Symbol} & \textbf{Definition} \\\hline
$f^{(s)}$ & The sample rate for time-slice $s$\\
$t^{(s)}$ & The time of ``late'' boundary for time-slice $s$\\
$N^{(s)}$ & The number of samples in time-slice $s$\\
$L^{(s)}$ & The number of principal components in time-slice $s$\\
$N^\uparrow$ & The number of sample points in the interpolation filter\\
$N^\downarrow$ & The number of sample points in the decimation filter\\
$M$ & The number of templates\\
\hline
\end{tabular}
\caption{constants}
\end{table}

\end{document}
