\documentclass[letterpaper,11pt]{article}
\usepackage{amssymb}
\usepackage{amsmath}
\usepackage{stmaryrd}  % gives \shortuparrow and \shortdownarrow

\title{Notation proposal}
\author{Leo and Nick}

\begin{document}
\maketitle

Definition of matched filter:
\begin{equation}
\rho_i [k] = \sum_{n=0}^{N-1} h_{i}[n] x [k-n]
\end{equation}
where $h_i[k]$ denotes the $k^{\mathrm{th}}$ sample of the $i^{\mathrm{th}}$
template waveform and $x[k-n]$ represents the $(k-n)^{\mathrm{th}}$ sample of
the data.

Definition of time slices:
\begin{align}
h_{i}[k] &= \sum_s h_{i}^{s}[k] \\
h_{i}^{s}[k] &= \left\{ \begin{aligned}
    h_{i}[k] & \; \text{if } t^{s+1} < \frac{k}{f} \leq t^{s} \\
    0 & \; \text{, otherwise }
    \end{aligned} \right. \\
h_{i}^{s}[k^{s}] &\equiv h_{i}^{s}\!\left[k\frac{f}{f^{s}}\right]
\end{align}

Throughout, we will use the convention that $k$ indexes time-series sampled at
the full sample rate $f$ and $k^s$ indexes time-series sampled at a reduced
sample rate $f^s$, ($k^{s} \equiv k f / f^{s}$).  For the case of band limited
functions it is sufficient to simply remove sample points to define the
downsampled version as is shown in the previous equation.  

In general one requiries an anti aliasing filter in order to decimate the data.
The definition of decimation for a function $p$:
\begin{equation}
p[k^{s+1}] = \sum_{n=-N^\shortdownarrow/2}^{N^\shortdownarrow/2-1} b^\shortdownarrow[n] \, p\!\left[k^{s}-n\right]
\end{equation}

It is also necessary to increas the sample rate of a function.  Definition of
interpolation for a function $p$:
\begin{equation}
p[k^{s}] = \sum_{n=-N^\shortuparrow/2}^{N^\shortuparrow/2-1} b^\shortuparrow[n] \, p\!\left[k^{s+1}-n\right]
%p[k^{s}] = \sum_{n=k^{s}-N^\shortuparrow/2}^{k^{s} + N^\shortuparrow/2-1} b^\shortuparrow[k^{s} - n] \, p\!\left[\frac{n f^{s+1}}{f^{s}}\right]
%f[k^{s}] = \sum_{n=-N^\shortuparrow/2}^{N^\shortuparrow/2-1} b^\shortuparrow[n] f\!\left[n \frac{f^{s}}{f^{s+1}}  \right]
\end{equation}

\textsc{svd}:
\begin{equation}
h^{s}_{i}\!\left[k^{s}\right] = \sum_{m=0}^{M-1} v^{s}_{im} \sigma^{s}_m u^{s}_{m}[k^{s}] \approx \sum_{l=0}^{L^s-1} v^{s}_{il} \sigma^{s}_l u^{s}_{l}[k^{s}]
\end{equation}

Early-warning \textsc{snr}:
\newcommand{\mystrut}{\rule[-2em]{0pt}{0pt}}  % line up the underbraces
\begin{equation}
    \rho^{s}_i[k^{s}] = \underbrace{\mystrut\sum_{l=0}^{L^{s}-1} v^{s}_{il} \sigma^{s}_l}_{\substack{\text{Reconstruction}\\ \text{with {\sc SVD}}}}
    \underbrace{\mystrut\sum_{n=0}^{N^{s} - 1} u^{s}_{l}[n] x[k^{s}-n]}_{\substack{\text{orthogonal}\\ \text{{\sc FIR} filter}}} + 
    % + 
    % \underbrace{\mystrut\sum_{n=-N^\shortuparrow/2}^{N^\shortuparrow/2-1} b^\shortuparrow[n] \sum_{q \in \mathbb{Z}} \delta\!\left[k-n, q \frac{f^{s}}{f^{s+1}}\right]}_\text{interpolation filter} 
    \underbrace{\mystrut\rho^{s+1}_i[k^{s}]}_{\substack{\text{accum.}\\ \text{prior {\sc snr}}}}
\end{equation}

\begin{table}
\begin{tabular}{rl}
\hline
\textbf{Symbol} & \textbf{Definition} \\\hline
$f^{s}$ & The sample rate for time-slice $s$\\
$t^{s}$ & The time of ``late'' boundary for time-slice $s$\\
$N^{s}$ & The number of samples in time-slice $s$\\
$L^{s}$ & The number of principal components in time-slice $s$\\
$N^\shortuparrow$ & The number of sample points in the interpolation filter\\
$N^\shortdownarrow$ & The number of sample points in the decimation filter\\
$M$ & The number of templates\\
\hline
\end{tabular}
\caption{constants}
\end{table}

\end{document}
