\section{Results}
\label{SECIV}\label{sec:results}

We generated receiver operating characteristic (\textsc{roc}) curves for a selection of choices for the composite detection statistic threshold and the \textsc{svd} tolerance.  For each \textsc{roc} curve, or `run', we analyzed mock data with injections to assess detection efficiencies and mock data without injections to evaluate false alarm rates.

\subsection{Detector noise characteristics}

We tested the new detection method with mock Advanced \textsc{ligo} data having a power spectrum prescribed by the ``zero detuning, high power'' noise model in \cite{Shoemaker:2009p9770}.  Colored Gaussian noise is generated by passing 5 independent realizations of white Gaussian noise sampled at 16384 Hz through a bank of 5 third-order \textsc{iir} filters, then summing the filters' outputs.  This carefully designed filter bank reproduces the noise model very faithfully, but since it is composed of \textsc{iir} filters it can produce mock data in realtime very cheaply.  See \ref{appendix:mock-data} for implementation details. \editorial{In this section, we refer to aLIGO run parameters, but the results are actually using initial LIGO noise and initial LIGO parameters.  This will be fixed.}

\subsection{Injection population}

For each run, we performed about $10^4$ injections into our mock dataset.  Injections were 2PN post-Newtonian inspiral waveforms with component masses independently and identically distributed in $[1, 3]$ $M_\odot$.  Injections were tapered \editorial{Check --taper-injection option with Drew.} at a frequency of 10~Hz where they enter the Advanced \textsc{ligo} detection band.  This dictated that the longest injection waveform had a duration of {\color{red} 1~777~s}.  In order to guarantee that no two injection waveforms overlapped in the data, the coalescence times of injections were spaced apart by {\color{red} 2~000~$\pm$~100~s}.  For the sake of economizing disk space, we simultaneously analyzed {\color{red} 16 statistically independent sets} of about $10^3$ injections apiece in the same {\color{red} 14 day} stretch of mock strain data.

Injections were distributed uniformly in log distance.  We used the following heuristic to provide a reasonably balanced number of distant, marginally detectable signals, and nearby, easily detected signals:
$$
d_\mathrm{min} = \frac{1}{2} \cdot \frac{\rho_\mathrm{candle}}{8.0 \cdot \min {d_\mathrm{eff}}}, \quad
d_\mathrm{max} = \frac{1}{4} \cdot \frac{\rho_\mathrm{candle}}{5.5 \cdot \max {d_\mathrm{eff}}}.
$$
Here, $\rho_\mathrm{candle}$ is a fiducially selected \textsc{snr}, and $d_\mathrm{eff}$ is the distance at which a face-on-system would produce a matched filter \textsc{snr} of $\rho_\mathrm{candle}$.

Orientations of injection sources are drawn uniformly, such that the sky location is distributed uniformly in $4\pi$ and the cosine of the binary inclination angle $\iota$ is distributed uniformly in $[0, \pi]$.

\begin{itemize}
\item 1 day of simulated $h(t)$.
\item $h(t)$ has power spectrum \sout{prescribed for ``zero detuning, high power'' model in \cite{Shoemaker:2009p9770}} that somewhat resembles initial \textsc{ligo} noise models
\item $h(t)$ generated by passing white, Gaussian noise through a bank of \textsc{iir} filters
\item 1 noninjections run
\item 10 injections runs
\item injections are distributed uniformly in log distance, uniformly in sky location and binary orientation
\item injections are reweighted to be uniformly distributed in volume
\item injections are 80$\pm$20 seconds apart
\item there are $\approx$ 10k injections
\end{itemize}

\begin{figure}[htbp]
\begin{center}
\includegraphics[scale=0.4]{figures/roc_99.pdf}
\includegraphics[scale=0.4]{figures/roc_9995.pdf}
\caption{Receiver operating characteristic (\textsc{roc}) curve of detection efficiency (\textsc{eff}) versus false alarm probability (\textsc{fap}).}
\label{fig:roc}
\end{center}
\end{figure}
