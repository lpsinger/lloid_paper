\section{Implementation}
\label{sec:implementation}

In this section we describe an implementation of the \lloid\ method described
in section \ref{sec:method} suitable for rapid \GW{}
searches for \CBC{}s.  The \lloid\ method requires several
computations that can be completed before the analysis is underway.  Thus
we divide the procedure into two stages 1) an off-line planning stage and 2) an
online, low-latency filtering stage.  The off-line stage can be done before the
analysis is started and updated asynchronously, whereas the online stage must
keep up with the detector output and produce search results as rapidly as
possible.  In the next two subsections we describe what these stages entail.

\subsection{Planning stage}

The planning stage begins with choosing templates that cover the space of
source parameters with a hexagonal grid~\citep{PhysRevD.76.102004} in order to
satisfy a minimum match criterion.  This assures a prescribed maximum loss in
\SNR\ for signals that fall between the chosen templates.  Typically the
minimum match is 0.97 corresponding to a maximum mismatch of 0.03.  Next, the
templates are subdivided into groups of neighbors called \emph{sub-banks} that
are appropriately sized so that each bank can be efficiently handled by a
single computer.  The neighbors are chosen to have comparable chirp mass, which
produces sub-banks with similar time-frequency evolution.  Dividing the source
parameter space into smaller sub-banks reduces the computational cost of the
\SVD{} and is the approach considered
in~\citep{Cannon:2010p10398}.  We choose time-slice boundaries as in
equation~\eqref{eq:time-sliced-templates} such that all of the templates within
a sub-bank are sub-critically sampled at progressively lower sample rates.
Next, the templates within the sub-bank are realized as \fir\ filter
coefficients.  For each time slice, the templates are down-sampled to the
appropriate sample rate.  Finally, the \SVD\ is applied to each time slice in
the sub-bank in order to produce a set of orthogonal \fir\ filters and a
reconstruction matrix that maps them back to the original templates as
described in equation~\eqref{eq:svddecomp}.  The down-sampled orthogonal \fir\
filter coefficients, the reconstruction matrix, and the time-slice boundaries
are all saved to disk.

\subsection{Filtering stage}

The \lloid\ algorithm could be used in a true sample-in-sample-out real-time
system.  However, such a system would likely require integration directly into
the data acquisition and storage system of the \GW{}
observatories.  A slightly more modest goal is to leverage existing low-%
latency, but not real-time, signal processing infrastructure in order to
implement the \lloid\ algorithm.  For the near-term this should be a viable 
solution for searches with order seconds of intrinsic latency.

We have implemented a prototype of the low-latency filtering stage using an
open-source signal processing environment called
\gstreamer\footnote{\url{http://gstreamer.net/}}.
\gstreamer\ is a vital component of many Linux systems, providing media
playback, authoring, and streaming on devices from cell phones to desktop
computers to streaming media servers.  Given the similarities of
\GW{} detector data to audio data it is not surprising that
\gstreamer\ is useful for our purpose. \gstreamer\ also provides some useful
stock signal processing elements such as re-samplers and filters.  We have
extended the \gstreamer\ framework by developing a library called
\gstlal\footnote{\url{https://www.lsc-group.phys.uwm.edu/daswg/projects/gstlal.html}}
that provides elements for \GW{} data analysis.

\subsubsection{Decimation}

The whitened detector data is reduced to successively
lower sample rates by decimation. Decimation involves applying an anti-aliasing
filter to the data, and then down-sampling by deleting samples.  We use a
192-tap \fir\ decimator provided by \gstreamer{}'s {\tt audioresample}
element.  The detector data is provided at every power-of-two sample rate
required by the template time slices described in \eqref{eq:time-slices}. Next,
these decimated data streams are fed into parallel \fir\ filter banks.

\subsubsection{\fir\ filters}

The \fir\ filtering is implemented using a \gstlal\ element called {\tt
lal\_firbank}, which produces $N$ channels of filter output from an $N\times M$ matrix of
\fir\ filter coefficients, representing the \SVD{} basis filters in our application.  This element is used in parallel branches in the
pipeline to implement the \SVD\ basis filters in each time
slice.  Rather than implement the time-sliced templates as zero-padded \fir\
filters as described in \eqref{eq:time-slices} we instead implement them as shorter
filters that contain only the nonzero samples.  Adding the appropriate time
offset to the filter output later in the pipeline makes up for the lack of
explicit zero-padding.

\subsubsection{Reconstruction}

From the outputs of the \SVD\ basis filters, we form the partial \SNR{}s for each time 
slice by multiplying by the reconstruction matrices.  This is accomplished by connecting 
the \texttt{gstlal} element \texttt{lal\_matrixmixer} to the output of
\texttt{lal\_firbank}.

\subsubsection{Interpolation}

In order to form the early-warning \SNR\ from each time slice, we have to add
the partial \SNR\ to the early-warning \SNR\ from the subsequent time slices.
If the next time slice does not have the same sample rate, its output must
first be interpolated.  This is done using the same \gstreamer\ element as was
used for decimation, {\tt audioresample}.  

\subsubsection{\SNR\ accumulation}

The early-warning \SNR\ for each time slice is formed by accumulating
interpolated early-warning \SNR\ from the subsequent time slice.  This process
continues until we have worked our way to the \SNR\ of the original templates
at the full sample rate $f^0$.  In this way, the \lloid\ algorithm and this
implementation naturally leads to a simple early-warning pipeline.
